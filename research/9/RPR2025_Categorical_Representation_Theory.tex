%%%%%%%%%
% SETUP %
%%%%%%%%%

\documentclass{beamer}
\mode<presentation> {
	\usetheme{Madrid}
}
\setbeamertemplate{frametitle continuation}[from second][]

\usepackage{graphicx}
\usepackage{booktabs}

\usepackage{tikz}
\usepackage{tikz-cd}
\usetikzlibrary{knots}
\usetikzlibrary{arrows.meta} 

% Title.
\title[Real Groups and Categorical Rep Theory]{Real Groups and Categorical Representation Theory}
\author{Daniel Dunmore}
\institute[UNSW]{
	University of New South Wales \\
	\medskip
	\textit{d.dunmore@unsw.edu.au}
}
\date{February 28, 2025}

% Bibliography.
\usepackage[%
	backend = biber,
	% BibLaTeX-Math package: https://github.com/konn/biblatex-math
	style = math-alphabetic,
	giveninits = true,
	dashed = false,
	url = false,
	doi = false,
	%sorting = none,
	minalphanames = 3,
	maxalphanames = 4
]{biblatex}
% Use the default font size for the bibliography.
\renewcommand*{\bibfont}{\normalsize}
% Use title case rather than sentence case for references.
\DeclareFieldFormat{titlecase}{#1}
% Put last names before first names.
\DeclareNameAlias{default}{family-given}
% Used for articles with appendices written by other authors.
\NewBibliographyString{bywithappendix}
\DefineBibliographyStrings{english}{
	bywithappendix = {with an appendix by}
}
% Specify the bibliography data file to use.
\addbibresource{References.bib}
% Sort by order present in .bib file.
%\nocite{*}


%%%%%%%%%%%%
% NOTATION %
%%%%%%%%%%%%

\usepackage{mathrsfs}

% Adds integral notation like \oiint.
\usepackage{esint}
% Blackboard bold symbols.
\usepackage{bbm}

\usepackage{accents}
% Tilde notation for vectors.
\newcommand{\ut}[1]{\underaccent{\tilde}{#1}}
% Arrow notation for vectors.
\usepackage{harpoon}
% Dirac bra-ket notation for quantum states.
\usepackage{braket}

% Differential formatting.
\usepackage{ifthen}
\usepackage{etoolbox}
\newcommand*{\ndiff}[1]{\mathrm{d}#1}
\newcommand*{\sdiff}[1]{\mathop{}\!\ndiff{#1}}
\newcommand{\rdiff}[3][]{
	\ifthenelse{\equal{#1}{}}
	{\frac{\mathrm{d}#2}{\mathrm{d}#3}}
	{\frac{\mathrm{d}^{#1}#2}{\forcsvlist\ndiff{#3}}}
}
\newcommand*{\npiff}[1]{\mathrm{\partial}#1}
\newcommand*{\spiff}[1]{\mathop{}\!\npiff{#1}}
\newcommand{\rpiff}[3][]{
	\ifthenelse{\equal{#1}{}}
	{\frac{\mathrm{\partial}#2}{\mathrm{\partial}#3}}
	{\frac{\mathrm{\partial}^{#1}#2}{\forcsvlist\npiff{#3}}}
}
% Inexact differential for physics.
\newcommand*{\dbar}[1]{\mathop{}\!\mathrm{\dj}#1}

% Metrics, inner products and norms.
\usepackage{mathtools}
\DeclarePairedDelimiter{\abs}{\lvert}{\rvert}
\DeclarePairedDelimiter{\inprod}{\langle}{\rangle}
\DeclarePairedDelimiter{\norm}{\lVert}{\rVert}
% This is used if we want an empty norm. 
\newcommand{\blank}{{}\cdot{}}

% Function notation.
\newcommand{\id}{\mathrm{id}}
\newcommand{\coker}{\mathrm{Coker}}
\newcommand{\im}{\mathrm{Im}}
\newcommand{\ev}{\mathrm{ev}}
\newcommand{\coev}{\mathrm{coev}}

% Category theory notation.
%\newcommand{\obset}{\textnormal{Ob}_{#1}\!\left(#2\right)}
%\newcommand{\morset}[2][]{\textnormal{Mor}_{#1}\!\left(#2\right)}
%\newcommand{\homset}[2][]{\textnormal{Hom}_{#1}\!\left(#2\right)}
%\newcommand{\End}[2][]{\textnormal{End}_{#1}\!\left(#2\right)}
%\newcommand{\opp}[1]{{#1}^{\textnormal{op}}}
%\newcommand{\textcat}[1]{\textnormal{\textsf{#1}}}
\newcommand{\obset}{\mathrm{Ob}}
\newcommand{\morset}{\mathrm{Mor}}
\newcommand{\homset}{\mathrm{Hom}}
\newcommand{\End}{\mathrm{End}}
\newcommand{\opp}{\mathrm{op}}
\newcommand{\textcat}[1]{\mathrm{\textsf{#1}}}
\newcommand{\textobj}[1]{\mathrm{\texttt{#1}}}

% Special notation.
%\newcommand{\chr}{\textnormal{char}}
%\newcommand{\Tr}{\textnormal{Tr}}
%\newcommand{\trv}{\textnormal{tr}}
%\newcommand{\Dim}{\textnormal{Dim}}
%\newcommand{\FPdim}{\textnormal{FPdim}}
\newcommand{\chr}{\mathrm{char}}
\newcommand{\Tr}{\mathrm{Tr}}
\newcommand{\trv}{\mathrm{tr}}
\newcommand{\Dim}{\mathrm{Dim}}
\newcommand{\FPdim}{\mathrm{FPdim}}

% Hiragana "yo" for the Yoneda embeddings.
\newcommand{\yo}{\text{\usefont{U}{min}{m}{n}\symbol{'210}}}
\DeclareFontFamily{U}{min}{}
\DeclareFontShape{U}{min}{m}{n}{<-> udmj30}{}

% Representation theory notation.
\newcommand{\Sym}{\mathrm{Sym}}
\newcommand{\Alt}{\mathrm{Alt}}

\usepackage{calc}

\newcommand*{\emphasis}[1]{\textcolor{structure}{\em #1}}

\usefonttheme[onlymath]{serif}

\DeclareCiteCommand{\aycite}
{\boolfalse{citetracker}\boolfalse{pagetracker}}
{\printtext[bibhyperref]{\printnames{labelname}\addcomma\addspace\printfield{year}}}
{\multicitedelim}
{}

% Negative horizontal phantom.
\newcommand{\nhphantom}[1]{\sbox0{#1}\hspace{-\the\wd0}}

\newtheorem{theoremdefinition}{Theorem-Definition}

%\usepackage{stmaryrd}

\renewcommand*{\bibfont}{\normalfont\footnotesize}

\setlength{\leftmargini}{30pt}


%%%%%%%%%%%%
% DOCUMENT %
%%%%%%%%%%%%

\begin{document}

%%%%%%%%%%%
% Prelude %
%%%%%%%%%%%

\begin{frame}
\noindent\\[-20pt]
\begin{figure}[!ht]
\titlepage
\noindent\\[-20pt]
\hspace{3.5cm}\hfill\includegraphics[width=2cm]{unsw-crest}\hfill\raisebox{-0.7093cm}{\includegraphics[width=3.5cm]{QR}}
\end{figure}
\end{frame}

\begin{frame}
\frametitle{Overview}
The theme of this talk will be using categorification to study the representation theory of real Lie groups.\\[2ex]

The material is quite dense; as a result, I would like to take a ``hands-on'' approach during this talk. I will be leaving out precise definitions of Soergel bimodules and birepresentations, and instead focusing on telling a coherent story that conveys where I fit into the picture, as well as the flavour of what I'm doing.
\end{frame}

\begin{frame}
\frametitle{Roadmap}
\begin{center}
\begin{minipage}{\widthof{(1) Real Groups and Lusztig--Vogan Modules}}
\setlength{\parskip}{4ex}
\tableofcontents
\end{minipage}
\end{center}
\end{frame}

%%%%%%%%%%%%%%%%%%%%%%%%%%%%%%%%%%%%%%%%%%
% Real Groups and Lusztig--Vogan Modules %
%%%%%%%%%%%%%%%%%%%%%%%%%%%%%%%%%%%%%%%%%%

\section{Real Groups and Lusztig--Vogan Modules}

\begin{frame}
\noindent\centerline{\LARGE\textcolor{structure}{\underline{Real Groups and Lusztig--Vogan Modules}}}
\end{frame}

\begin{frame}
\frametitle{Motivation I}
Physics is, (very) roughly speaking, the study of symmetry and symmetry breaking.\\[2ex]

For our purposes, this is captured by the study of Lie groups and their representations.\\[2ex]

The primary motivation for this talk will be the classification of unitary representations of real Lie groups.\\[2ex]

Understanding the entire class is hopeless for now, so we instead look at the more manageable class of \textcolor{structure}{admissible representations of real reductive groups}.
\end{frame}

\begin{frame}
\frametitle{Motivation II}
For real reductive groups in the so-called \textcolor{structure}{Harish--Chandra class}, the irreducible admissible representations are classified by a certain set of parameters due to the \textcolor{structure}{Langlands classification}.\\[2ex]

This classification scheme is similar to other classification schemes in representation theory! That is, we have some standard principal series module, with every irreducible module contained uniquely within it.\\[2ex]

Given such a scheme, we're often interested in the relationship between the principal series and irreducibles.
\end{frame}

\begin{frame}
\frametitle{Motivation III}
To this end, Lusztig and Vogan created a family of modules over the Iwahori--Hecke algebra that are built from connected real reductive groups (\cite{Vog82}, \cite{LV83}).\\[2ex]

These modules admit a standard basis and a canonical basis, with the change-of-basis coefficients giving rise to a collection of polynomials.\\[2ex]

The values of these \textcolor{structure}{Kazhdan--Lusztig--Vogan polynomials} at $1$ determine the Jordan–H\"{o}lder multiplicities of irreducible submodules of the standard module for the underlying reductive group.\\[2ex]

The philosophy of categorification has been fruitful in similar contexts. What can it tell us here?
\end{frame}

%%%%%%%%%%%%%%%%%%%%%%%%%%%%%%%%%%%%
% Lusztig--Vogan Module Categories %
%%%%%%%%%%%%%%%%%%%%%%%%%%%%%%%%%%%%

\section{Lusztig--Vogan Module Categories}

\begin{frame}
\noindent\centerline{\LARGE\textcolor{structure}{\underline{Lusztig--Vogan Module Categories}}}
\end{frame}

\begin{frame}
\frametitle{The General Setup}
The Lusztig--Vogan (LV) module categories take as ingredients
\begin{itemize}
\item a connected, complex, reductive algebraic group $G$;
\item a Borel subgroup $B \subseteq G$;
\item a holomorphic involution $\theta : G \to G$;
\item a finite index subgroup $K \subseteq G^\theta$ \textit{(the identity component of $G^\theta$)};
\item a maximal torus $T_K \subseteq B \cap K$.\\
\end{itemize}
\begin{align*}
\begin{split}
\theta &\rightsquigarrow \text{real form of $G$}\\
T_K \subseteq G &\rightsquigarrow \text{Weyl group}\\
B &\rightsquigarrow \text{choice of simple roots}%\\[2ex]
\end{split}
\end{align*}
%The full construction of these categories is a bit involved.\\
%In the ``equal rank'' case -- when $T_K$ is Abelian -- it simplifies dramatically!
\end{frame}

\begin{frame}
\frametitle{The Equal Rank Case}
\begin{enumerate}
\item Let $W$ be a Weyl group with $S \subseteq W$ a choice of simple roots and $W_K \subseteq W$ a finite index subgroup.
\item Define a polynomial algebra $R \coloneqq \mathbbm{k}[\alpha_s : s \in S]$ graded in degree $2$.
\item We have a \emphasis{geometric action} of $W$ on $R$ given by\\[-4ex]
\begin{align*}
\begin{split}
s(\alpha_t) = \alpha_t + 2\cos\!\left(\frac{\pi}{m_{st}}\right)\!\alpha_s.\\[-1ex]
\end{split}
\end{align*}
%\item Let $R^{W_K}$ be the polynomials in $R$ that are invariant under $W_K$.
\item For any right coset $w \in W_K\backslash W$, let $R_w$ be the \emphasis{$w$-standard $(R^{W_K}, R)$-bimodule} given by $R$ as a vector space with the right action given by the $w$-twisting $p \cdot_w f \coloneqq pw(f)$, for $p \in R_w$, $f \in R$.
\end{enumerate}
\begin{definition}[\aycite{LR22}] The corresponding \emphasis{(equal rank) Lusztig--Vogan module category} is\\[-4ex]
\begin{align*}
\begin{split}
\mathcal{N}_{LV}^0 = \langle R_w \otimes_R X : w \in W_K\backslash W,\ X \in \obset(\mathbb{S}\textcat{Bim}(W, S))\rangle_{\oplus,\ominus,(1)}.
\end{split}
\end{align*}
\end{definition}
\end{frame}

\begin{frame}
\frametitle{Example: $\textup{SL}(2, \mathbb{R})$}
Possibly the simplest LV category is the one corresponding to $\textup{SL}(2, \mathbb{R})$.\\[2ex]
Our ingredients are
\begin{itemize}
\item $G = \textup{SL}(2, \mathbb{C})$,
\item $B = \left\{\!\begin{pmatrix}a&b\hphantom{{}^{-1}}\\0&a^{-1}\end{pmatrix} : a \in \mathbb{C}^\times, b \in \mathbb{C}\right\}$,
\item $\theta : \begin{pmatrix}a&b\\c&d\end{pmatrix} \mapsto \begin{pmatrix}\hphantom{-}a&-b\\-c&\hphantom{-}d\end{pmatrix}$,
\item $T_K = K = G^\theta = \left\{\!\begin{pmatrix}a&0\hphantom{{}^{-1}}\\0&a^{-1}\end{pmatrix} : a \in \mathbb{C}^\times\right\}$.\\[2ex]
\end{itemize}
This corresponds to $W = S_2 = \{1, s\}$, $S = \{s\}$ and $W_K = \{1\}$.\\[2ex]
The geometric action is given by $1(\alpha_t) = \alpha_t$ and $s(\alpha_t) = -\alpha_t$.
\end{frame}

\begin{frame}
\frametitle{Example: $\textup{SL}(2, \mathbb{R})$}
Let's compute the indecomposables of this category!\\[2ex]

We'll start by computing products of standard bimodules and indecomposables from $\mathbb{S}\textcat{Bim}$.
\begin{alignat*}{2}
R_1 \otimes_R B_1 &\cong R,&\qquad R_1 \otimes_R B_{s} &\cong B_{s},\\
R_{s} \otimes_R B_1 &\cong R_{s},&\qquad R_{s} \otimes_R B_{s} &\cong B_{s}.
\end{alignat*}
As it turns out, all of these are indecomposable in $\mathcal{N}_{LV}^0$, so luckily we don't need to look at direct summands!
\end{frame}

\begin{frame}[fragile]
\frametitle{Example: $\textup{SL}(2, \mathbb{R})$}
Here's something interesting we can observe.\\[2ex]

For $P, Q \in \textup{Ind}(\mathcal{N}_{LV}^0)/\!\sim$, write $Q \geq P$ if there exists $X \in \obset(\mathbb{S}\textcat{Bim})$ for which $Q$ is isomorphic to a direct summand of $P \otimes_R B$.\\[2ex]

Stepping back to our example, this gives us the cell structure
\begin{center}
\begin{tikzpicture}[baseline=(current bounding box.center)]
\draw (0,0.75) -- (2.5,0);
\draw (0,-0.75) -- (2.5,0);
\path[fill=white] (-0.375,-0.375-0.75) rectangle (0.6,0.375+0.75);
\path[fill=white] (2.5-0.6,-0.375) rectangle (2.875,0.375);
\draw[fill=red!10!blue!20!gray!10!white] (-0.375,-0.375+0.75) rectangle (0.375,0.375+0.75);
\node at (0,0.75) {$R$};
\draw[fill=red!10!blue!20!gray!10!white] (-0.375,-0.375-0.75) rectangle (0.375,0.375-0.75);
\node at (0,-0.75) {$R_s$};
\draw[fill=red!10!blue!20!gray!10!white] (2.5-0.375,-0.375) rectangle (2.875,0.375);
\node at (2.5,0) {$B_s$};
\end{tikzpicture}\ .
\end{center}
\end{frame}

\begin{frame}
\frametitle{Example: $\textup{SU}(2, 1)$}
Let's look at a more involved example.\\[2ex]
Our ingredients are
\begin{itemize}
\item $G = \textup{SL}(3, \mathbb{C})$,
\item $\theta : \begin{pmatrix}a&b&c\\d&e&f\\g&h&i\end{pmatrix} \mapsto \begin{pmatrix}\hphantom{-}a&\hphantom{-}b&-c\\\hphantom{-}d&\hphantom{-}e&-f\\-g&-h&\hphantom{-}i\end{pmatrix}$,
\item $K = G^\theta = \left\{\!\begin{pmatrix}a&b&0\\c&d&0\\0&0&e\end{pmatrix} : a, b, c, d, e \in \mathbb{C},\ (ad - bc)e = 1\right\}$,
\item $T_K = \left\{\!\begin{pmatrix}a&0&0\\0&b&0\\0&0&\frac{1}{ab}\end{pmatrix} : a, b \in \mathbb{C}^\times\right\}$.\\[2ex]
\end{itemize}
%This corresponds to $W = S_2 = \{1, s\}$, $S = \{s\}$ and $W_K = \{1\}$.\\[2ex]
%The geometric action is given by $1(\alpha_t) = \alpha_t$ and $s(\alpha_t) = -\alpha_t$.
\end{frame}

\begin{frame}
\frametitle{Example: $\textup{SU}(2, 1)$}
The indecomposables of this category are much more difficult to compute. With some work, however, it can be shown that they are
\begin{align*}
\begin{split}
R,\quad R_t,\quad R_{ts},\quad B_t,\quad R_t \otimes_R B_s,\quad R^s \otimes_{R^W} R.
\end{split}
\end{align*}
\noindent Products of standard bimodules and indecomposables are already no longer necessarily indecomposable!\\[2ex]
The cell structure is given by
\begin{center}
\begin{tikzpicture}[baseline=(current bounding box.center)]
\draw (0,0.75) -- (2.5,0);
\draw (0,-0.75) -- (2.5,0);
\draw (0,0.75) -- (-2.5,0);
\draw (0,-0.75) -- (-2.5,0);
\path[fill=white] (-0.6,-0.375-0.75) rectangle (0.6,0.375+0.75);
\path[fill=white] (-1.225,-0.375-0.75) rectangle (1.225,0.375-0.75);
\path[fill=white] (2.5-0.6,-0.375) rectangle (2.875,0.375);
\path[fill=white] (-2.5+0.6,-0.375) rectangle (-2.875,0.375);
\draw[fill=red!10!blue!20!gray!10!white] (-0.375,-0.375+0.75) rectangle (0.375,0.375+0.75);
\node at (0,0.75) {$R_t$};
\draw[fill=red!10!blue!20!gray!10!white] (-1,-0.375-0.75) rectangle (1,0.375-0.75);
\node at (0,-0.75) {$R^s \otimes_{R^W} R$};
\draw[fill=red!10!blue!20!gray!10!white] (2.5-0.375,-0.375) rectangle (2.875+2,0.375);
\node at (3.5,0) {$R_t \otimes_R B_s, R_{ts}$};
\draw[fill=red!10!blue!20!gray!10!white] (-2.5+0.375,-0.375) rectangle (-2.875-0.5,0.375);
\node at (-2.75,0) {$R, B_t$};
\end{tikzpicture}\ .
\end{center}
\end{frame}

%%%%%%%%%%%%%%%%%%%%%%%%%%%%%
% Weak Jordan-Holder Series %
%%%%%%%%%%%%%%%%%%%%%%%%%%%%%

\section{Weak Jordan--H\"{o}lder Series}

\begin{frame}
\noindent\centerline{\LARGE\textcolor{structure}{\underline{Weak Jordan--H\"{o}lder Series}}}
\end{frame}

\begin{frame}
\frametitle{Weak Jordan--H\"{o}lder Series}
In \cite[Theorem 8]{MM16}, Mazorchuk and Miemietz developed a higher categorical analogue to the Jordan--H\"{o}lder theorem.\\[2ex]
This is central to higher category theory, as it shows how different birepresentations decompose into simple (transitive) birepresentations.\\[2ex]
It's quite technical, so let's apply it to one of our examples.
\end{frame}

\begin{frame}
\frametitle{Example: $\textup{SL}(2, \mathbb{R})$}
Our filtrations consist of directed order ideals of $\textup{Ind}(\mathcal{N}_{LV}^0)/\!\sim$.\\ Each ideal has one less object than the next.\\[2ex]

In the $\textup{SL}(2, \mathbb{R})$ case, one such filtration is
\begin{align*}
\begin{split}
Q_1 \coloneqq \{[B_{s}]\} \subset Q_2 \coloneqq \{[R_{s}], [B_{s}]\} \subset Q_3 \coloneqq \{[R], [R_{s}], [B_{s}]\}.
\end{split}
\end{align*}
These give rise to the additive $\mathbb{S}\textcat{Bim}$-module subcategories
\begin{align*}
\begin{split}
\mathcal{M}_{Q_1} = \langle B_{s} \rangle_{\oplus,(1)},\quad\mathcal{M}_{Q_2} = \langle R_{s}, B_{s}\rangle_{\oplus,(1)},\quad\mathcal{M}_{Q_3} = \langle R, R_{s}, B_{s}\rangle_{\oplus,(1)}.
\end{split}
\end{align*}
Denote by $\mathcal{M}_{Q_i/Q_{i-1}}$ the quotient of $\mathcal{M}_{Q_i}$ by the two-sided ideal generated by the identity morphisms in $Q_{i-1}$.
\end{frame}

\begin{frame}
\frametitle{Example: $\textup{SL}(2, \mathbb{R})$}
What are the simple module categories?\\[2ex]

Well, $\mathcal{M}_{Q_1}$ doesn't change.\\
The quotient functor from $\mathcal{M}_{Q_2}$ to $\mathcal{M}_{Q_2/Q_1}$ takes $B_s$ to $0$.\\
The quotient functor from $\mathcal{M}_{Q_3}$ to $\mathcal{M}_{Q_3/Q_2}$ takes $R_s$ and $B_s$ to $0$.\\[2ex]

As it happens, the functor sending $R_s$ to $R$ defines an equivalence of module categories $\mathcal{M}_{Q_2/Q_1} \simeq \mathcal{M}_{Q_3/Q_2}$.\\[2ex]

Don't be fooled by ``$R_s \otimes_R R_s \sim R = 0$''!\\[2ex]

These are transitive module categories, but there is a unique way of making them simple by throwing away non-invertible morphisms.
\end{frame}

%%%%%%%%%%%%%%%%
% What's Next? %
%%%%%%%%%%%%%%%%

\section{What's Next?}

\begin{frame}
\noindent\centerline{\LARGE\textcolor{structure}{\underline{What's Next?}}}
\end{frame}

\begin{frame}
\frametitle{What's Next?}
We would like to eventually look at these simple decompositions outside of the type $A$ case.\\[2ex]

The final goal is to consider a notion of extensions in order to build new module categories from such simple constituents.\\[2ex]

The hope is that these new module categories, may contain their own interesting representation theoretic data.
\end{frame}

%%%%%%%%%%%%%%
% References %
%%%%%%%%%%%%%%

\section*{References}

\begin{frame}[allowframebreaks]
\frametitle{References}
\printbibliography[heading = none]
\end{frame}

%%%%%%%%%%%%%
% Thank you %
%%%%%%%%%%%%%

\begin{frame}
\noindent\centerline{\LARGE\textcolor{structure}{\underline{Thank you very much for your attention!}}}
%\vspace*{2em}
%\centerline{\LARGE Any questions?}
\end{frame}

\end{document}