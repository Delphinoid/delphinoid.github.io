\input{preamble.tex}

\begin{document}

\thispagestyle{fancy}

\begin{center}
\LARGE\scshape Classification of Fusion Categories\noindent\\[-\linespacing]
\rule{0.75\linewidth}{1pt}
\end{center}
\noindent\\[-0.75\linespacing]

\ruledsection{Prologue}{1}
\noindent\\ What are fusion categories? What are near-groups, Haagerup--Izumi categories and quadratic\linebreak categories? What is modular data? What are $6j$ symbols? What is the even part of a subfactor?\newpage

\noindent Let $\mathcal{C}$ be a fusion category with representatives $\{X_i\}_{i \in \Gamma}$ of isomorphism classes of simple objects, and choose bases for each multiplicity space $H_{i,j}^l := \homset_\mathcal{C}(X_l, X_i \otimes X_j)$. The {\em (quantum) $6j$-symbols of $\mathcal{C}$} are the matrix blocks $\Phi_{i_1,i_2,i_3}^{i_4}$ of the change-of-basis matrices $\Phi_{i_1,i_2,i_3} \coloneqq \bigoplus_{i_4 \in \Gamma} \Phi_{i_1,i_2,i_3}^{i_4}$ given by
\begin{align*}
\begin{split}
\Phi_{i_1,i_2,i_3}^{i_4} := v^{-1}(X_{i_4}, X_{i_1}, X_{i_2} \otimes X_{i_3}) \circ (\yo(\alpha_{X_{i_1}, X_{i_2}, X_{i_3}}))_{X_{i_4}} \circ u(X_{i_4}, X_{i_1} \otimes X_{i_2}, X_{i_3}),
\end{split}
\end{align*}
\noindent where
\begin{align*}
\begin{split}
\Phi_{i_1,i_2,i_3}^{i_4} : \bigoplus_{j\in\Gamma} (H_{j,i_3}^{i_4} \otimes_\mathbbm{k} H_{i_1,i_2}^j) \to \bigoplus_{l\in\Gamma} (H_{i_1,l}^{i_4} \otimes_\mathbbm{k} H_{i_2,i_3}^l).
\end{split}
\end{align*}
\noindent Recall the Yoneda embedding
\begin{align*}
\begin{split}
\yo_*(X) &\coloneqq \homset_\mathcal{C}(-, X),\\
\yo_*(f : X \to Y) &\coloneqq \homset_\mathcal{C}(-, X) \Rightarrow \homset_\mathcal{C}(-, Y).
\end{split}
\end{align*}
\noindent Taking the Yoneda embedding of a component $\alpha_{X,Y,Z} : (X \otimes Y) \otimes Z \to X \otimes (Y \otimes Z)$ of the associativity natural isomorphism $\alpha$, we obtain a natural isomorphism
\begin{align*}
\begin{split}
\yo_*(\alpha_{X,Y,Z}) = \homset_\mathcal{C}(-, (X \otimes Y) \otimes Z) \Rightarrow \homset_\mathcal{C}(-, X \otimes (Y \otimes Z)).
\end{split}
\end{align*}
\noindent Thus we have an isomorphism of vector spaces
\begin{align*}
\begin{split}
[\yo_*(\alpha_{X,Y,Z})](W) : \homset_\mathcal{C}(W, (X \otimes Y) \otimes Z) \xrightarrow{\sim} \homset_\mathcal{C}(W, X \otimes (Y \otimes Z));
\end{split}
\end{align*}
\noindent that is, an invertible matrix. In other words, the associativity is given by matrices indexed by $X, Y, Z, W$. But why do we call these $6j$ symbols? Well, we can simplify our picture further. Suppose we have an isomorphism
\begin{align*}
\begin{split}
\homset_\mathcal{C}(X_4, (X_1 \otimes X_2) \otimes X_3) \xrightarrow{\sim} \homset_\mathcal{C}(X_4, X_1 \otimes (X_2 \otimes X_3)).
\end{split}
\end{align*}
\noindent Let $X_5$ and $X_6$ be simple summands of $(X_1 \otimes X_2)$ and $(X_2 \otimes X_3)$, respectively. Then we can determine our isomorphism by determining the matrices of the form
\begin{align*}
\begin{split}
\homset_\mathcal{C}(X_4, X_5 \otimes X_3) \to \homset_\mathcal{C}(X_4, X_1 \otimes X_6)
\end{split}
\end{align*}
\noindent for all such $X_5$ and $X_6$. These matrices are exactly the $6j$ symbols, where the six simple objects $X_1, X_2, X_3, X_4, X_5, X_6$ play the role of the ``six $j$'s''.
\noindent Note that these matrices are indeed parameterized by all ``six $j$'s'', as there could be many invertible matrices that give us a maps of the aforementioned form. We need $X_2$ in order to use the pentagon diagram for the associativity constraint and hence determine the specific invertible matrix corresponding to the associator. Moreover, $X_2$ tells us how the blocks
\begin{align*}
\begin{split}
\homset_\mathcal{C}(X_4, X_5 \otimes X_3) \to \homset_\mathcal{C}(X_4, X_1 \otimes X_6)
\end{split}
\end{align*}
\noindent fit together into the block diagonal matrix
\begin{align*}
\begin{split}
\homset_\mathcal{C}(X_4, (X_1 \otimes X_2) \otimes X_3) \xrightarrow{\sim} \homset_\mathcal{C}(X_4, X_1 \otimes (X_2 \otimes X_3)).
\end{split}
\end{align*}
\newpage

\ruledsection{The Cuntz Algebra Approach of Izumi}{2}
\noindent\\ Take $\textcat{Vec}_G$ to be skeletal. Consider an associativity constraint $a_{ghk} : ghk --> ghk$. Since $ghk$ is a simple object, $\homset(ghk, ghk) \cong \mathbbm{k}$, whence $a_{ghk} = \lambda_{ghk}\textup{id}_{ghk}$ for some $\lambda_{ghk} \in \mathbbm{k}^\times$. Note that the pentagon diagram enforces certain conditions on our choice of $\lambda_{ghk}$; in particular, if we look at this diagram, we'll see that $\lambda_{ghk} = \omega(g, h, k)$ for some $3$-cocycle $\omega$. By this, we mean a map $\omega : G \times G \times G \to \mathbbm{k}^\times$ satisfying
\begin{align*}
\begin{split}
\omega(x, y, zw)\omega(xy, z, w)\omega(y, z, w)^{-1}\omega(x, yz, w)^{-1}\omega(x, y, z)^{-1} = 1
\end{split}
\end{align*}
for all $x, y, z, w \in G$. We will henceforth denote by $\textcat{Vec}_G^\omega$ the category of $G$-graded vector spaces with associativity constraint $a_{ghk} = \omega(g, h, k)\textup{id}_{ghk}$, for all $g, h, k \in G$, and $\textcat{Vec}_G$ the category of $G$-graded vector spaces with trivial associativity.\\

\noindent Consider the category $\textup{End}(M)$, for $M$ a hyperfinite type III factor. This category is strict, as $\rho \otimes \sigma \coloneqq \rho \circ \sigma$ by definition. Every near-group category with group $G$ contains some copy of $\textcat{Vec}_G^\omega$ corresponding to the group-like part. Because every unitary near-group category is a subcategory of $\textup{End}(M)$ and is hence itself strict, we know that it will actually contain the ``strictification'' of some $\textcat{Vec}_G^\omega$. However, Izumi shows that if $\mathcal{C}$ is any fusion category containing a simple object that is fixed under tensor products with invertibles (that is, there exists some simple object $X$ such that $X \otimes g \cong X$ for all invertible $g$), then it contains a copy of $\textcat{Vec}_G$, for $G$ the group of isomorphism classes of invertible objects. He shows in addition that if the fusion category is also unitary, then $g \otimes X = X$ (but we may not necessarily have that $X \otimes g = X$). The upshot is that we almost know how objects are tensored, since the group-like part will have trivial associativity (that is, $g \otimes h = gh$). We just need to understand $X \otimes g$ and $X \otimes X$, as well as the morphisms.\\

\noindent In \cite{Izu17}, Izumi showed that every unitary near-group category $\mathcal{C}$ with multiplicity $m$ is equivalent to a subcategory of $\textup{End}(M)$, where $M$ is the hyperfinite type $\textup{III}_1$ factor. In particular, it is generated by a single irreducible endomorphism $\rho \in \textup{End}_0(M)$ satisfying the fusion rules
\begin{align*}
\begin{split}
[\rho] \otimes [\rho] = \bigoplus_{g \in G} [\alpha_g] \oplus [\rho]^{\oplus m},\\
[\alpha_g] \otimes [\alpha_h] = [\alpha_{gh}],\\
[\alpha_g] \otimes [\rho] = [\rho] \otimes [\alpha_g] = [\rho],\\
\end{split}
\end{align*}
where the map $\alpha : G \to \textup{Aut}(M)$ induces an injective homomorphism from $G$ into $\textup{Out}(M)$.\\

\noindent The main result of \cite{Izu17} is \cite[Theorem 4.9]{Izu17}. Essentially, there is a bijective correspondence between the set of equivalence classes of unitary near-group categories with finite group $G$ and multiplicity parameter $m$ and the set of equivalence classes of admissible tuples $(\mathcal{K}, j_1, j_2, V, U_\mathcal{K}, \chi, l)$ (see \cite[Definition 4.8]{Izu17}). Here $\mathcal{K}$ is the finite-dimensional Hilbert space $\homset(\rho, \rho^2)$, $j_1$ and $j_2$ are two antilinear isometries of $\mathcal{K}$, $V$ and $U_\mathcal{K}$ are unitary representations of $G$ on $\mathcal{K}$, $\{\chi_g\}_{g \in G}$ are characters of $G$ and $l$ is a linear map from $\mathcal{K}$ to the set $\mathcal{B}(\mathcal{K}, \mathcal{K} \otimes \mathcal{K})$ of bounded operators $\mathcal{K} \to \mathcal{K} \otimes \mathcal{K}$.\newpage

\noindent By \cite[Theorem 9.1]{Izu17}, the unitary near-group categories with finite Abelian group $G$ and $m = \abs{G}$ are completely classified tuples of the form $(\langle\cdot,\cdot\rangle, a, b, c)$, where $\langle\cdot,\cdot\rangle : G \times G \to \mathbb{T}$ is a non-degenerate symmetric bicharacter and where $a : G \to \mathbb{T}$, $b : G \to \mathbb{T}$ and $c \in \mathbb{T}$ satisfy various conditions. When we say that $\langle\cdot,\cdot\rangle$ is a bicharacter, we mean that
\begin{align*}
\begin{split}
\langle xy, z\rangle = \langle x, z\rangle\langle y, z\rangle\qquad\text{and}\qquad\langle x, yz\rangle = \langle x, y\rangle\langle x, z\rangle
\end{split}
\end{align*}
for all $x, y, z \in G$. By non-degenerate, we mean that
\begin{align*}
\begin{split}
\langle x, \cdot\rangle = \langle y, \cdot\rangle
\end{split}
\end{align*}
\noindent if and only if $x = y$. This is equivalent to the map $\varphi : G \to \homset(G, \mathbb{T})$ given by $x \mapsto \langle x, \cdot\rangle$ being an isomorphism.\\

\noindent\begin{definition}\textup{(Cuntz Algebra).} Let $\{S_i\}_{i=1}^n$ be a set of isometries on an infinite-dimensional Hilbert space $\mathscr{H}$. Suppose moreover that these isometries satisfy the {\em Cuntz relation}
\begin{align*}
\begin{split}
\sum_{k=1}^n S_kS_k^* = 1.
\end{split}
\end{align*}
\noindent The {\em Cuntz algebra $\mathcal{O}_n$} is the universal $C^*$-algebra $C^*(S_1, \dots, S_n)$.\\
\end{definition}

\noindent\begin{remark} Note that, as isometries, $S_i^*S_i = 1$. In particular, we must have that $S_i^*S_j = \delta_{i,j}$ for all $i, j \in \{1, \dots, n\}$. This follows from the fact that a sum of projections is itself a projection if and only if the projections in the sum are pairwise orthogonal. The Cuntz relation is essentially ensuring that the sum of the projections $S_iS_i^*$ is the trivial projection.\\
\end{remark}

\noindent 
\newpage

\noindent\begin{example}\textup{(Fibonacci Category).} Let's look at the Fibonacci category. This is the near-group with $G = \{0\}$ and $m = 1$. Our choice for $\langle\cdot,\cdot\rangle$ is obvious, and \cite[Lemma 7.1]{Izu17} tells us that
\begin{align*}
\begin{split}
c^3 a(0) = \csqrt{n} = 1 \implies a(0) = c^{-3}.
\end{split}
\end{align*}
\noindent Moreover, \cite[Theorem 9.1]{Izu17} tells us that $b$ is defined by $b : 0 \mapsto -1/d$, where $d$ corresponds to the dimension of our irreducible generator $\rho$. Let's determine $c$ and $d$. Because $b$ is equal to its own Fourier transform, \cite[Theorem 9.1]{Izu17} tells us that
\begin{align*}
\begin{split}
b(0) = ca(0)b(0) \implies a(0) = c^{-1}.
\end{split}
\end{align*}
\noindent In order for $c^{-1} = c^{-3}$, we require $c = \pm 1$. Finally, \cite[Equation 9.5]{Izu17} tells us that
\begin{align*}
\begin{split}
b(0)b(0)b(0) &= b(0)b(0) \mp \frac{1}{d},
\end{split}
\end{align*}
\begin{align*}
\begin{split}
\implies &-\frac{1}{d^3} = \frac{1}{d^2} \mp \frac{1}{d},\\
\implies &\pm d^2 - d - 1 = 0.
\end{split}
\end{align*}
\noindent This only has a real solution when $c = 1$, whence $d$ is nothing but the golden ratio (as it cannot be negative in the unitary case - this alternative solution, known as the Galois dual, corresponds to a non-unitary near-group in this case and many others). This is exactly what we would expect, as $d$ is the dimension of $X$ (where $d^2 = 1 + d$ comes from the fusion rule $X^2 = \mathbbm{1} \oplus X$).\\
\end{example}

\noindent\begin{example}\textup{($G = \mathbb{Z}/2\mathbb{Z}$).} Let's look at the case where $G = \mathbb{Z}/2\mathbb{Z}$ and $m = 2$. This near-group corresponds to the even part of the type $A_4$ subfactor. We know the dimension is
\begin{align*}
\begin{split}
d_\pm \coloneqq \frac{m \pm \csqrt{m^2 + 4n}}{2} = 1 \pm \csqrt{3}.
\end{split}
\end{align*}
\noindent In the unitary setting, we of course ask that $d$ be positive, and hence we choose $d = d_+$. The only possibility for a non-degenerate bicharacter is
\begin{align*}
\begin{split}
\langle 0, 0\rangle = 1,\qquad\langle 0, 1\rangle = \langle 1, 0\rangle = 1\qquad\text{and}\qquad\langle 1, 1\rangle = -1.
\end{split}
\end{align*}
\noindent From \cite[Equation 7.8]{Izu17}, it follows that
\begin{align*}
\begin{split}
a(0) = 1\qquad\text{and}\qquad a(1) = \pm i.
\end{split}
\end{align*}
\noindent Meanwhile, \cite[Equation 9.4]{Izu17} tells us that
\begin{align*}
\begin{split}
\overline{b(1)} = \pm i b(1) \implies \mathfrak{R}(b(1)) = \mp \mathfrak{I}(b(1)),
\end{split}
\end{align*}
\noindent whence \cite[Equation 9.3]{Izu17} gives us
\begin{align*}
\begin{split}
\mathfrak{R}(b(1))^2 + \mathfrak{I}(b(1))^2 = (b(1)\overline{b(1)})^2 = \frac{1}{2} \implies b(1) = \frac{1 - a(1)}{2}.
\end{split}
\end{align*}
\noindent It then follows from evaluating \cite[Equation 9.1]{Izu17} with $g = 0$ and rearranging for $c$ that
\begin{align*}
\begin{split}
c = \frac{1 - \csqrt{3} + a(1)(1 + \csqrt{3})}{2\csqrt{2}}.
\end{split}
\end{align*}
\noindent Note that we may choose either $a(1) = i$ or $a(1) = -i$; both of these lead to solutions. Moreover, in the non-unitary setting, we may take the Galois conjugate of $d$.
\end{example}
\newpage

\noindent\begin{example}\label{UHIZ2}\textup{($G = \mathbb{Z}/2\mathbb{Z}$).} Let's determine the Haagerup--Izumi categories with $G = \mathbb{Z}/2\mathbb{Z}$. Let
\begin{align*}
\begin{split}
d_\pm \coloneqq \frac{n \pm \csqrt{n^2 + 4}}{2},
\end{split}
\end{align*}
\noindent where in this example $d \coloneqq 1 + \csqrt{2}$. Izumi's classification involves a triplet $(\epsilon_h(g), \omega(g), A_{h,k}(g))$, where $\epsilon_h(g) \in \{-1, 1\}$, $\omega(g) \in \mathbb{T}$ and $A_{h,k}(g) \in \mathbb{C}$ satisfy \cite[Equations 4.1--4.9]{Izu18}. Well, we know
\begin{align*}
\begin{split}
\epsilon_0(0) = \epsilon_1(0) = 1\qquad\text{and}\qquad\epsilon_0(1) = \epsilon_0(1)\epsilon_0(1) \implies \epsilon_0(1) = 1.
\end{split}
\end{align*}
\noindent By \cite[Equation 4.7]{Izu18},
\begin{align*}
\begin{split}
A_{0,0}(g) = A_{0,0}(g)\omega(g),
\end{split}
\end{align*}
\noindent which tells us that either $\omega(g) = 1$ or $A_{0,0}(g) = 0$ for each $g \in G$. Let's fix any $g \in G$ and consider the case when $A_{0,0}(g) = 0$. In this case, however, \cite[Equations 4.3 and 4.4]{Izu18} give us
\begin{align*}
\begin{split}
A_{1,0}(g)\overline{A_{\delta_{g,0}-g,0}(g)} = 1 - \frac{\abs{\omega(g)}}{d} \implies \text{``}\frac{1}{d^2} = 1 - \frac{1}{d}\text{''}.
\end{split}
\end{align*}
\noindent This ``equality'' is nonsense; we must therefore have $\omega(g) = 1$ for all $g \in G$. Suppose now that $\epsilon_1(1) = 1$. Then \cite[Equation 4.7]{Izu18} gives us
\begin{align*}
\begin{split}
A_{0,1}(0) = A_{1,1}(0) = A_{1,0}(0)\qquad\text{and}\qquad A_{0,1}(1) = A_{1,1}(1) = A_{1,0}(1),
\end{split}
\end{align*}
\noindent while \cite[Equation 4.8]{Izu18} gives us $A_{1,1}(0) = A_{1,1}(1)$. Now, \cite[Equations 4.4 and 4.6]{Izu18} tell us
\begin{align*}
\begin{split}
A_{0,1}(0) A_{1,1}(1) + A_{1,1}(0) A_{1,0}(1) = 0.
\end{split}
\end{align*}
\noindent Thus $A_{0,1}(g) = A_{1,1}(g) = A_{1,0}(g) = 0$ and hence $A_{0,0}(g) = -1/d$ by \cite[Equation 4.3]{Izu18}. However, in this case we cannot satisfy \cite[Equation 4.9]{Izu18}. Suppose instead that $\epsilon_1(1) = -1$. With this new $2$-cocycle, \cite[Equation 4.7]{Izu18} now gives us
\begin{align*}
\begin{split}
A_{0,1}(0) = A_{1,1}(0) = A_{1,0}(0)\qquad\text{and}\qquad A_{0,1}(1) = -A_{1,1}(1) = A_{1,0}(1),
\end{split}
\end{align*}
\noindent while \cite[Equation 4.8]{Izu18} gives us $A_{1,1}(1) = -A_{1,1}(0)$. We then see by \cite[Equation 4.4]{Izu18} that
\begin{align*}
\begin{split}
A_{0,1}(0) A_{1,0}(0) + A_{1,1}(0) A_{1,1}(0) = 1 \implies A_{1,0}(0) = \pm\frac{1}{\csqrt{2}} = \pm\frac{1}{d-1},
\end{split}
\end{align*}
\noindent and by \cite[Equation 4.9]{Izu18} that
\begin{align*}
\begin{split}
A_{0,0}(0)A_{1,0}(0)^2 = A_{1,0}(0)^2 + A_{1,0}(0)^3 \implies A_{0,0}(0) = 1 + A_{1,0}(0) = \frac{d-1 \pm 1}{d-1}.
\end{split}
\end{align*}
\noindent Finally, \cite[Equation 4.3]{Izu18} allows us to deduce
\begin{align*}
\begin{split}
A_{1,0}(0) = -\frac{1}{d-1},
\end{split}
\end{align*}
\noindent whence
\begin{align*}
\begin{split}
A(0) = \frac{1}{d-1}\begin{pmatrix}d-2 & -1\\ -1 & -1\end{pmatrix}\qquad\text{and}\qquad A(1) = \frac{1}{d-1}\begin{pmatrix}d-2 & -1\\ -1 & \hphantom{-}1\end{pmatrix}.
\end{split}
\end{align*}
\noindent This category is nothing but the even part of the type $A_7$ subfactor.\\
\end{example}
\newpage

\noindent\begin{remark} Suppose that $\abs{G}$ is odd. Then \cite[Equation 4.1]{Izu18} tells us that $\epsilon_h(g) = 1$, while \cite[Equation 4.2]{Izu18} tells us that $\omega(g)$ does not depend on $g$. Moreover, $A_{h,k}(g)$ cannot depend on $g$ by \cite[Equation 4.5]{Izu18}, and either $\omega = 1$ or $A_{0,0} = 0$ by \cite[Equation 4.7]{Izu18}. In this case, \cite[Equations 4.1--4.9]{Izu18} reduce to the following four equations.
\begin{align*}
\begin{split}
A_{h,k} = A_{-k,h-k}\omega = A_{k-h,-h}\overline{\omega},
\end{split}
\end{align*}
\begin{align*}
\begin{split}
\sum_{h \in G} A_{h,0} = -\frac{\overline{\omega}}{d_\pm},
\end{split}
\end{align*}
\begin{align*}
\begin{split}
\sum_{h \in G} A_{h-g,k} A_{k,h-g'} = \delta_{g,g'} - \frac{\delta_{k,0}}{d_\pm},
\end{split}
\end{align*}
\begin{align*}
\begin{split}
\sum_{l \in G} A_{x+y,l} A_{-x,l+p} A_{-y,l+q} = A_{p+x,q+x+y} A_{q+y,p+x+y} - \frac{\delta_{x,0}\delta_{y,0}}{d_\pm}.
\end{split}
\end{align*}
\end{remark}
\noindent The first three equations above are precisely \cite[Equations 4.7, 4.8 and 4.9]{EG17}! In particular, to see that our third equation is equivalent to \cite[Equation 4.9]{EG17}, we simply make the change of variables $\hat{g} \coloneqq g' - g$ and $\hat{h} \coloneqq h - g'$, whence we obtain
\begin{align*}
\begin{split}
\sum_{\hat{h} \in G} A_{\hat{h}+\hat{g},k} A_{k,\hat{h}} = \delta_{\hat{g},0} - \frac{\delta_{k,0}}{d_\pm}.
\end{split}
\end{align*}
\noindent Similarly, using our first equation while making the change of variables $\hat{l} \coloneqq l-x-y$, $\hat{p} \coloneqq p+x+y$, $\hat{q} \coloneqq q+x+y$, $\hat{x} \coloneqq -x$ and $\hat{y} \coloneqq -y$, our fourth equation becomes
\begin{align*}
\begin{split}
\overline{\omega}\sum_{\hat{l} \in G} A_{\hat{l},\hat{x}+\hat{y}} A_{\hat{x},\hat{l}+\hat{p}} A_{\hat{y},\hat{l}+\hat{q}} = A_{\hat{y}+\hat{p},\hat{q}} A_{\hat{x}+\hat{q},\hat{p}} - \frac{\delta_{\hat{x},0}\delta_{\hat{y},0}}{d_\pm},
\end{split}
\end{align*}
\noindent showing that it is equivalent to \cite[Equation 4.11]{EG17}.
\newpage

\ruledsection{The Leavitt Algebra Approach of Evans--Gannon}{3}
\noindent\\ The important result is \cite[Theorem 2]{EG17}.\\

\noindent\begin{definition}\textup{(Leavitt Algebra).} Let $X \coloneqq (x_{ij})$ and $Y \coloneqq (y_{ij})$ be $m \times n$ and $n \times m$ matrices of symbols, respectively. The {\em Leavitt $K$-algebra of type $(m, n)$} is the free associative unital $K$-algebra
\begin{align*}
\begin{split}
\mathcal{L}_K(m, n) \coloneqq \frac{K[x_{ij}, y_{ij}]}{\langle XY = I_m, YX = I_n\rangle}.
\end{split}
\end{align*}
\noindent In other words, it is the universal $K$-algebra with generators
\begin{align*}
\begin{split}
\{x_{ij} : 1 \leq i \leq m, 1 \leq j \leq n\} \sqcup \{y_{ij} : 1 \leq i \leq n, 1 \leq j \leq m\}
\end{split}
\end{align*}
\noindent and {\em Leavitt--Cuntz relations}
\begin{align*}
\begin{split}
\sum_{k=1}^m y_{ik} x_{kj} = \delta_{i,j}\qquad\text{and}\qquad\sum_{k=1}^n x_{ik} y_{kj} = \delta_{i,j},
\end{split}
\end{align*}
\noindent for all suitable $i, j$.\\
\end{definition}

\noindent Consider the Leavitt $\mathbb{C}$-algebra of type $(1, n)$, which we shall henceforth denote by $\mathcal{L}_n \coloneqq \mathcal{L}_\mathbb{C}(1, n)$. We have that $\mathcal{O}_n = C^*(\mathcal{L}_n)$, where $x_i = S_i$ and $y_i = S_i^*$. Let's think about what this means precisely. The Leavitt--Cuntz relations for $m = 1$ become
\begin{align*}
\begin{split}
y_i x_j = \delta_{i,j}\qquad\text{and}\qquad\sum_{k=1}^n x_k y_k = 1.
\end{split}
\end{align*}
\noindent We may endow $\mathcal{L}_n$ with the structure of a $*$-algebra by defining a conjugate homogeneous\linebreak antihomomorphism that sends $x_i \mapsto y_i$ and $y_i \mapsto x_i$. We further define
\begin{align*}
\begin{split}
\norm{a} \coloneqq \sup\{p(a) : \textup{$p$ is a $C^*$-seminorm on $\mathcal{L}_n$}\}
\end{split}
\end{align*}
\noindent for all $a \in \mathcal{L}_n$, where a $C^*$-seminorm is just a seminorm for which $p(a^*a) = p(a)^2$ and $p(ab) \leq p(a)p(b)$. Note that $0 \leq \norm{a} \leq 1$, as $1 = \norm{y_ix_i} = \norm{x_i^2}$ and hence $\norm{x_i} = \norm{y_i} = 1$ for all $i$. The condition $p(ab) \leq p(a)p(b)$ ensures that $\mathcal{I} \coloneqq \{a \in \mathcal{L}_n : \norm{a} = 0\}$ is an ideal in $\mathcal{L}_n$. Our $C^*$-seminorm then descends to a $C^*$-norm on the quotient $\mathcal{L}_n/\mathcal{I}$. The completion of $\mathcal{L}_n/\mathcal{I}$ with respect to this $C^*$-norm is known as the universal $C^*$-algebra of $\mathcal{L}_n$, denoted by $C^*(\mathcal{L}_n)$. This is precisely $\mathcal{O}_n$ by definition. We may therefore view $\mathcal{L}_n$ as the polynomial part of $\mathcal{O}_n$.
\newpage

\begin{example}\textup{(Yang--Lee Category).} Let $G = \{0\}$. Then \cite[Equation 4.7]{EG17} demands that
\begin{align*}
\begin{split}
A_{0,0} = \omega A_{0,0} = \overline{\omega}A_{0,0} \implies \omega = 1,
\end{split}
\end{align*}
whence \cite[Equation 4.8]{EG17} tells us that
\begin{align*}
\begin{split}
A_{0,0} = -\frac{1}{d_\pm}.
\end{split}
\end{align*}
\noindent The rest of \cite[Equations 4.7--4.10]{EG17} are satisfied by these choices. Hence by \cite[Theorem 2]{EG17}, we have two fusion categories for $G = \{0\}$; a unitary one with $\pm = +$ (the Fibonacci category) and a non-unitary one with $\pm = -$ (the Yang--Lee category).\\
\end{example}

\noindent\begin{example}\textup{($G = \mathbb{Z}/2\mathbb{Z}$).} The equations we must satisfy for $\abs{G}$ even are given in \cite{Izu18} \textcolor{red}{(is this true?)}. Adapting our argument from \hyperref[UHIZ2]{Example \ref*{UHIZ2}}, we see that there is exactly one non-unitary Haagerup--Izumi category with $G = \mathbb{Z}/2\mathbb{Z}$. This corresponds to $\epsilon_h(g) = (-1)^{gh}$, $\omega(g) = 1$,
\begin{align*}
\begin{split}
A(0) = \frac{1}{d-1}\begin{pmatrix}d & 1\\ 1 & 1\end{pmatrix}\qquad\text{and}\qquad A(1) = \frac{1}{d-1}\begin{pmatrix}d & \hphantom{-}1\\ 1 & -1\end{pmatrix}.
\end{split}
\end{align*}
\end{example}

\newpage

\renewcommand\thesection{R}
\ruledsectionstar{References}{References}
\begingroup
\setlength{\emergencystretch}{.5em}
\printbibliography[heading=none]
\endgroup
\newpage

\end{document}