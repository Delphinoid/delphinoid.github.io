\input{preamble.tex}

\begin{document}

\thispagestyle{fancy}

\begin{center}
\LARGE\scshape Classification of Fusion Categories\noindent\\[-\linespacing]
\rule{0.75\linewidth}{1pt}
\end{center}
\noindent\\[-0.75\linespacing]

\ruledsection{Prologue}{1}
\noindent\\ What are fusion categories? What are near-groups?\newpage

\ruledsection{The Cuntz Algebra Approach of Izumi}{2}
\noindent\\ Take $\textcat{Vec}_G$ to be skeletal. Consider an associativity constraint $a_{ghk} : ghk --> ghk$. Since $ghk$ is a simple object, $\homset(ghk, ghk) \cong \mathbbm{k}$, whence $a_{ghk} = \lambda_{ghk}\textup{id}_{ghk}$ for some $\lambda_{ghk} \in \mathbbm{k}^\times$. Note that the pentagon diagram enforces certain conditions on our choice of $\lambda_{ghk}$; in particular, if we look at this diagram, we'll see that $\lambda_{ghk} = \omega(g, h, k)$ for some $3$-cocycle $\omega$. By this, we mean a map $\omega : G \times G \times G \to \mathbbm{k}^\times$ satisfying
\begin{align*}
\begin{split}
\omega(x, y, zw)\omega(xy, z, w)\omega(y, z, w)^{-1}\omega(x, yz, w)^{-1}\omega(x, y, z)^{-1} = 1
\end{split}
\end{align*}
for all $x, y, z, w \in G$. We will henceforth denote by $\textcat{Vec}_G^\omega$ the category of $G$-graded vector spaces with associativity constraint $a_{ghk} = \omega(g, h, k)\textup{id}_{ghk}$, for all $g, h, k \in G$, and $\textcat{Vec}_G$ the category of $G$-graded vector spaces with trivial associativity.\\

\noindent Consider the category $\textup{End}(M)$, for $M$ a hyperfinite type III factor. This category is strict, as $\rho \otimes \sigma \coloneqq \rho \circ \sigma$ by definition. Every near-group category with group $G$ contains some copy of $\textcat{Vec}_G^\omega$ corresponding to the group-like part. Because every unitary near-group category is a subcategory of $\textup{End}(M)$ and is hence itself strict, we know that it will actually contain the ``strictification'' of some $\textcat{Vec}_G^\omega$. However, Izumi shows that if $\mathcal{C}$ is any fusion category containing a simple object that is fixed under tensor products with invertibles (that is, there exists some simple object $X$ such that $X \otimes g \cong X$ for all invertible $g$), then it contains a copy of $\textcat{Vec}_G$, for $G$ the group of isomorphism classes of invertible objects. He shows in addition that if the fusion category is also unitary, then $g \otimes X = X$ (but we may not necessarily have that $X \otimes g = X$). The upshot is that we almost know how objects are tensored, since the group-like part will have trivial associativity (that is, $g \otimes h = gh$). We just need to understand $X \otimes g$ and $X \otimes X$, as well as the morphisms.\\

\noindent In \cite{Izu17}, Izumi showed that every unitary near-group category $\mathcal{C}$ with multiplicity $m$ is equivalent to a subcategory of $\textup{End}(M)$, where $M$ is the hyperfinite type $\textup{III}_1$ factor. In particular, it is generated by a single irreducible endomorphism $\rho \in \textup{End}_0(M)$ satisfying the fusion rules
\begin{align*}
\begin{split}
[\rho] \otimes [\rho] = \bigoplus_{g \in G} [\alpha_g] \oplus [\rho]^{\oplus m},\\
[\alpha_g] \otimes [\alpha_h] = [\alpha_{gh}],\\
[\alpha_g] \otimes [\rho] = [\rho] \otimes [\alpha_g] = [\rho],\\
\end{split}
\end{align*}
where the map $\alpha : G \to \textup{Aut}(M)$ induces an injective homomorphism from $G$ into $\textup{Out}(M)$.\\

\noindent The main result of \cite{Izu17} is \cite[Theorem 4.9]{Izu17}. Essentially, there is a bijective correspondence between the set of equivalence classes of unitary near-group categories with finite group $G$ and multiplicity parameter $m$ and the set of equivalence classes of admissible tuples $(\mathcal{K}, j_1, j_2, V, U_\mathcal{K}, \chi, l)$ (see \cite[Definition 4.8]{Izu17}). Here $\mathcal{K}$ is the finite-dimensional Hilbert space $\homset(\rho, \rho^2)$, $j_1$ and $j_2$ are two antilinear isometries of $\mathcal{K}$, $V$ and $U_\mathcal{K}$ are unitary representations of $G$ on $\mathcal{K}$, $\{\chi_g\}_{g \in G}$ are characters of $G$ and $l$ is a linear map from $\mathcal{K}$ to the set $\mathcal{B}(\mathcal{K}, \mathcal{K} \otimes \mathcal{K})$ of bounded operators $\mathcal{K} \to \mathcal{K} \otimes \mathcal{K}$.\newpage

\noindent By \cite[Theorem 9.1]{Izu17}, the unitary near-group categories with finite Abelian group $G$ and $m = \abs{G}$ are completely classified tuples of the form $(\langle\cdot,\cdot\rangle, a, b, c)$, where $\langle\cdot,\cdot\rangle : G \times G \to \mathbb{T}$ is a non-degenerate symmetric bicharacter and where $a : G \to \mathbb{T}$, $b : G \to \mathbb{T}$ and $c \in \mathbb{T}$ satisfy various conditions. When we say that $\langle\cdot,\cdot\rangle$ is a bicharacter, we mean that
\begin{align*}
\begin{split}
\langle xy, z\rangle = \langle x, z\rangle\langle y, z\rangle\qquad\text{and}\qquad\langle x, yz\rangle = \langle x, y\rangle\langle x, z\rangle
\end{split}
\end{align*}
for all $x, y, z \in G$. By non-degenerate, we mean that
\begin{align*}
\begin{split}
\langle x, \cdot\rangle = \langle y, \cdot\rangle
\end{split}
\end{align*}
\noindent if and only if $x = y$. This is equivalent to the map $\varphi : G \to \homset(G, \mathbb{T})$ given by $x \mapsto \langle x, \cdot\rangle$ being an isomorphism.\\
\newpage

\noindent\begin{example}\textup{(Fibonacci Category).} Let's look at the Fibonacci category. This is the near-group with $G = \{0\}$ and $m = 1$. Our choice for $\langle\cdot,\cdot\rangle$ is obvious, and \cite[Lemma 7.1]{Izu17} tells us that
\begin{align*}
\begin{split}
c^3 a(0) = \csqrt{n} = 1 \implies a(0) = c^{-3}.
\end{split}
\end{align*}
\noindent Moreover, \cite[Theorem 9.1]{Izu17} tells us that $b$ is defined by $b : 0 \mapsto -1/d$, where $d$ corresponds to the dimension of our irreducible generator $\rho$. Let's determine $c$ and $d$. Because $b$ is equal to its own Fourier transform, \cite[Theorem 9.1]{Izu17} tells us that
\begin{align*}
\begin{split}
b(0) = ca(0)b(0) \implies a(0) = c^{-1}.
\end{split}
\end{align*}
\noindent In order for $c^{-1} = c^{-3}$, we require $c = \pm 1$. Finally, \cite[Equation 9.5]{Izu17} tells us that
\begin{align*}
\begin{split}
b(0)b(0)b(0) &= b(0)b(0) \mp \frac{1}{d},\\
\implies -\frac{1}{d^3} &= \frac{1}{d^2} \mp \frac{1}{d},\\
\implies \pm d^2 - d - 1 = 0.
\end{split}
\end{align*}
\noindent This only has a real solution when $c = 1$, whence $d$ is nothing but the golden ratio (as it cannot be negative in the unitary case - this alternative solution, known as the Galois dual, corresponds to a non-unitary near-group in this case and many others). This is exactly what we would expect, as $d$ is the dimension of $X$ (where $d^2 = 1 + d$ comes from the fusion rule $X^2 = \mathbbm{1} \oplus X$).\\
\end{example}

\noindent\begin{example}\textup{($G = \mathbb{Z}/2\mathbb{Z}$).} Let's look at the case where $G = \mathbb{Z}/2\mathbb{Z}$ and $m = 2$. This near-group corresponds to the even part of the type $A_4$ subfactor. We know the dimension is
\begin{align*}
\begin{split}
d_\pm \coloneqq \frac{m \pm \csqrt{m^2 + 4n}}{2} = 1 \pm \csqrt{3}.
\end{split}
\end{align*}
\noindent In the unitary setting, we of course ask that $d$ be positive, and hence we choose $d = d_+$. The only possibility for a non-degenerate bicharacter is
\begin{align*}
\begin{split}
\langle 0, 0\rangle = 1,\qquad\langle 0, 1\rangle = \langle 1, 0\rangle = 1\qquad\text{and}\qquad\langle 1, 1\rangle = -1.
\end{split}
\end{align*}
\noindent From \cite[Equation 7.8]{Izu17}, it follows that
\begin{align*}
\begin{split}
a(0) = 1\qquad\text{and}\qquad a(1) = \pm i.
\end{split}
\end{align*}
\noindent Meanwhile, \cite[Equation 9.4]{Izu17} tells us that
\begin{align*}
\begin{split}
\overline{b(1)} = \pm i b(1) \implies \mathfrak{R}(b(1)) = \mp \mathfrak{I}(b(1)),
\end{split}
\end{align*}
\noindent whence \cite[Equation 9.3]{Izu17} gives us
\begin{align*}
\begin{split}
\mathfrak{R}(b(1))^2 + \mathfrak{I}(b(1))^2 = (b(1)\overline{b(1)})^2 = \frac{1}{2} \implies b(1) = \frac{1 - a(1)}{2}.
\end{split}
\end{align*}
\noindent It then follows from evaluating \cite[Equation 9.1]{Izu17} with $g = 0$ and rearranging for $c$ that
\begin{align*}
\begin{split}
c = \frac{1 - \csqrt{3} + a(1)(1 + \csqrt{3})}{2\csqrt{2}}.
\end{split}
\end{align*}
\noindent Note that we may choose either $a(1) = i$ or $a(1) = -i$; both of these lead to solutions. Moreover, in the non-unitary setting, we may take the Galois conjugate of $d$.
\end{example}
\newpage

\ruledsection{The Leavitt Algebra Approach of Evans--Gannon}{3}
\noindent\\ .\newpage

\renewcommand\thesection{R}
\ruledsectionstar{References}{References}
\begingroup
\setlength{\emergencystretch}{.5em}
\printbibliography[heading=none]
\endgroup
\newpage

\end{document}