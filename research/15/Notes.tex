\input{preamble.tex}

\begin{document}

\thispagestyle{fancy}

\begin{center}
\LARGE\scshape Classification of Fusion Categories\noindent\\[-\linespacing]
\rule{0.75\linewidth}{1pt}
\end{center}
\noindent\\[-0.75\linespacing]

\ruledsection{Prologue}{1}
\noindent\\ What are fusion categories? What are near-groups, Haagerup--Izumi categories and quadratic\linebreak categories? What is modular data? What are $6j$ symbols? What is the even part of a subfactor?\newpage

\noindent Let $\mathcal{C}$ be a skeletal fusion category over $\mathbbm{k}$. Recall the Yoneda embedding
\begin{align*}
\begin{split}
\yo_*(X) \coloneqq \homset_\mathcal{C}(-, X)\qquad\text{and}\qquad\yo_*(f : X \to Y) \coloneqq (\homset_\mathcal{C}(-, X) \Rightarrow \homset_\mathcal{C}(-, Y)).
\end{split}
\end{align*}
\noindent Taking the Yoneda embedding of a component $\alpha_{X,Y,Z} : (X \otimes Y) \otimes Z \to X \otimes (Y \otimes Z)$ of the associativity natural isomorphism $\alpha$, we obtain a natural isomorphism
\begin{align*}
\begin{split}
\yo_*(\alpha_{X,Y,Z}) = (\homset_\mathcal{C}(-, (X \otimes Y) \otimes Z) \Rightarrow \homset_\mathcal{C}(-, X \otimes (Y \otimes Z))).
\end{split}
\end{align*}
\noindent Thus we have an isomorphism of vector spaces
\begin{align*}
\begin{split}
[\yo_*(\alpha_{X,Y,Z})](W) : \homset_\mathcal{C}(W, (X \otimes Y) \otimes Z) \xrightarrow{\sim} \homset_\mathcal{C}(W, X \otimes (Y \otimes Z));
\end{split}
\end{align*}
\noindent that is, an invertible matrix. In other words, the associativity is given by matrices indexed by $X, Y, Z, W$. But we can simplify things further! Suppose we now choose representatives $\{X_i\}_{i \in \Gamma}$ of isomorphism classes of simple objects and fix a choice of basis (gauge) for each {\em multiplicity space} $H_{i,j}^h := \homset_\mathcal{C}(X_h, X_i \otimes X_j)$. Then by semisimplicity,
\begin{align*}
\begin{split}
X_i \otimes X_j \cong \bigoplus_{h \in \Gamma} X_h^{\dim_\mathbbm{k}(H_{i,j}^h)},
\end{split}
\end{align*}
\noindent whence we have both
\begin{align*}
\begin{split}
\bigoplus_{m\in\Gamma} H_{m,i_3}^{i_0} \otimes_\mathbbm{k} H_{i_1,i_2}^m &= \bigoplus_{m\in\Gamma} \homset_\mathcal{C}(X_{i_0}, X_m \otimes X_{i_3}) \otimes_\mathbbm{k} \homset_\mathcal{C}(X_m, X_{i_1} \otimes X_{i_2})\\
&\cong \homset_\mathcal{C}(X_{i_0}, (X_{i_1} \otimes X_{i_2}) \otimes X_{i_3})
\end{split}
\end{align*}
\noindent and
\begin{align*}
\begin{split}
\bigoplus_{n\in\Gamma} H_{i_1,n}^{i_0} \otimes_\mathbbm{k} H_{i_2,i_3}^n &= \bigoplus_{n\in\Gamma} \homset_\mathcal{C}(X_{i_0}, X_{i_1} \otimes X_n) \otimes_\mathbbm{k} \homset_\mathcal{C}(X_n, X_{i_2} \otimes X_{i_3})\\
&\cong \homset_\mathcal{C}(X_{i_0}, X_{i_1} \otimes (X_{i_2} \otimes X_{i_3})).
\end{split}
\end{align*}
\noindent We thus obtain canonical bases for the above Hom-spaces. The {\em (quantum) $6j$-symbols of $\mathcal{C}$} are then defined to be the matrix blocks of the form
\begin{align*}
\begin{split}
\Phi_{i_1,i_2,i_3}^{i_0,m,n} : H_{m,i_3}^{i_0} \otimes_\mathbbm{k} H_{i_1,i_2}^m \to H_{i_1,n}^{i_0} \otimes_\mathbbm{k} H_{i_2,i_3}^n
\end{split}
\end{align*}
\noindent such that the block-diagonal matrices $\Phi_{i_1,i_2,i_3}^{i_0} := \bigoplus_{m\in\Gamma}\bigoplus_{n\in\Gamma} \Phi_{i_1,i_2,i_3}^{i_0,m,n}$ give a change-of-basis
\begin{align*}
\begin{split}
\Phi_{i_1,i_2,i_3}^{i_0} : \homset_\mathcal{C}(X_{i_0}, (X_{i_1} \otimes X_{i_2}) \otimes X_{i_3}) \cong \homset_\mathcal{C}(X_{i_0}, X_{i_1} \otimes (X_{i_2} \otimes X_{i_3}))%\xrightarrow{\sim} \homset_\mathcal{C}(X_{i_0}, X_{i_1} \otimes (X_{i_2} \otimes X_{i_3}))
\end{split}
\end{align*}
\noindent between the canonical bases. These matrices $\Phi_{i_1,i_2,i_3}^{i_0}$ themselves form the blocks of a block-diagonal matrix $\Phi_{i_1,i_2,i_3} \coloneqq \bigoplus_{i_0 \in \Gamma} \Phi_{i_1,i_2,i_3}^{i_0}$, which gives us a change-of-basis $(X_{i_1} \otimes X_{i_2}) \otimes X_{i_3} \cong X_{i_1} \otimes (X_{i_2} \otimes X_{i_3})$ describing how the associator acts on each summand $X_{i_0}$. As it happens, we in fact have that%\xrightarrow{\sim} X_{i_1} \otimes (X_{i_2} \otimes X_{i_3})$ describing how the associator acts on each summand $X_{i_0}$. As it happens, we in fact have that
\begin{align*}
\begin{split}
\Phi_{i_1,i_2,i_3}^{i_0,m,n} = v_n^{-1}(X_{i_0}, X_{i_1}, X_{i_2} \otimes X_{i_3}) \circ [\yo_*(\alpha_{X_{i_1}, X_{i_2}, X_{i_3}})]({X_{i_0}}) \circ u_m(X_{i_0}, X_{i_1} \otimes X_{i_2}, X_{i_3}),
\end{split}
\end{align*}
\noindent where
\begin{align*}
\begin{split}
[u_m(W, U, V)](f \otimes_\mathbbm{k} g) \coloneqq (g \otimes \id_V) \circ f\quad\text{and}\quad [v_n(W, U, V)](f' \otimes_\mathbbm{k} g') \coloneqq (\id_U \otimes g') \circ f'
\end{split}
\end{align*}
\noindent for all $f \in \homset_\mathcal{C}(W, X_m \otimes V)$, $g \in \homset_\mathcal{C}(X_m, U)$, $f' \in \homset_\mathcal{C}(W, U \otimes X_n)$ and $g \in \homset_\mathcal{C}(X_n, V))$.\\
\newpage

\noindent More explicitly, suppose we have an isomorphism
\begin{align*}
\begin{split}
\homset_\mathcal{C}(X_4, (X_1 \otimes X_2) \otimes X_3) \cong \homset_\mathcal{C}(X_4, X_1 \otimes (X_2 \otimes X_3)).%\xrightarrow{\sim} \homset_\mathcal{C}(X_4, X_1 \otimes (X_2 \otimes X_3)).
\end{split}
\end{align*}
\noindent Let $X_5$ and $X_6$ be simple summands of $(X_1 \otimes X_2)$ and $(X_2 \otimes X_3)$, respectively. Then we can determine our isomorphism by determining the block matrices of the form
\begin{align*}
\begin{split}
\homset_\mathcal{C}(X_4, X_5 \otimes X_3) \to \homset_\mathcal{C}(X_4, X_1 \otimes X_6)
\end{split}
\end{align*}
\noindent for all such $X_5$ and $X_6$. These matrices are exactly the $6j$ symbols, where the six simple objects $X_1, X_2, X_3, X_4, X_5, X_6$ play the role of the eponymous ``six $j$'s''. This is also where $u$ and $v$ come in; given $f \in H_{5,3}^4$ and $g \in H_{1,2}^5$, we have $[u_5(X_4, X_1 \otimes X_2, X_3)](f \otimes_\mathbbm{k} g) : X_4 \to (X_1 \otimes X_2) \otimes X_3$, and similarly $[v_6(X_4, X_1, X_2 \otimes X_3)](f' \otimes_\mathbbm{k} g') : X_4 \to X_1 \otimes (X_2 \otimes X_3)$ for $f' \in H_{1,6}^4$ and $g' \in H_{2,3}^6$.\\

%\noindent Note that these matrices are indeed parameterized by all ``six $j$'s'', as there could be many invertible matrices that give us a maps of the aforementioned form. We need $X_2$ in order to use the pentagon diagram for the associativity constraint and hence determine the specific invertible matrix\linebreak corresponding to the associator. Moreover, $X_2$ tells us how the blocks
%\begin{align*}
%\begin{split}
%\homset_\mathcal{C}(X_4, X_5 \otimes X_3) \to \homset_\mathcal{C}(X_4, X_1 \otimes X_6)
%\end{split}
%\end{align*}
%\noindent fit together into the block diagonal matrix
%\begin{align*}
%\begin{split}
%\homset_\mathcal{C}(X_4, (X_1 \otimes X_2) \otimes X_3) \xrightarrow{\sim} \homset_\mathcal{C}(X_4, X_1 \otimes (X_2 \otimes X_3)).
%\end{split}
%\end{align*}
%\noindent\\[-\linespacing]

\noindent\begin{example}\textup{($\textcat{Vec}_{\mathbb{Z}/2\mathbb{Z}}$).} Consider $\textcat{Vec}_{\mathbb{Z}/2\mathbb{Z}}$, a category with two isomorphism classes of simple objects -- $[\mathbbm{1}]$ and $[X]$ -- satisfying the fusion rule
\begin{align*}
\begin{split}
[X] \otimes [X] = [\mathbbm{1}].
\end{split}
\end{align*}
\noindent Let's assume that it's skeletal -- that is, $X \otimes X = \mathbbm{1}$ -- and start by computing the change-of-basis matrix $\Phi_{X,X,X}$. This matrix can be written in block diagonal form as
\begin{align*}
\begin{split}
\Phi_{X,X,X} = \begin{pmatrix}\Phi_{X,X,X}^\mathbbm{1}&0\\0&\Phi_{X,X,X}^X\end{pmatrix}.
\end{split}
\end{align*}
\noindent First, observe that
\begin{align*}
\begin{split}
\Phi_{X,X,X}^\mathbbm{1} : \homset(\mathbbm{1}, (X \otimes X) \otimes X) \to \homset(\mathbbm{1}, X \otimes (X \otimes X));
\end{split}
\end{align*}
\noindent but $(X \otimes X) \otimes X = X = X \otimes (X \otimes X)$, so $\Phi_{X,X,X}^\mathbbm{1}$ is just the isomorphism of $0$-dimensional vector spaces; that is, $\Phi_{X,X,X}^\mathbbm{1} = 0$. Let's now compute $\Phi_{X,X,X}^X$. %Denote $X \otimes X$ by $\widetilde{\mathbbm{1}}$.
We shall start by choosing bases for the multiplicity spaces $H_{X,X}^{{\mathbbm{1}}}$, $H_{{\mathbbm{1}},X}^X$ and $H_{X,{\mathbbm{1}}}^X$. These spaces are all $1$-dimensional, so we will choose basis elements $\iota_{X,X}^{{\mathbbm{1}}}$, $\lambda_X^{-1}$ and $\rho_X^{-1}$, respectively. Here we have a canonical choice of basis elements for $H_{{\mathbbm{1}},X}^X$ and $H_{X,{\mathbbm{1}}}^X$; namely, the left and right unitors $\lambda$ and $\rho$. This culminates in the basis elements
\begin{align*}
\begin{split}
\iota_{(X \otimes X) \otimes X}^X \coloneqq (\iota_{X,X}^{{\mathbbm{1}}} \otimes \id_X) \circ \lambda_X^{-1} = [u_{{\mathbbm{1}}}(X, X \otimes X, X)](\lambda_X^{-1} \otimes_\mathbbm{k} \iota_{X,X}^{{\mathbbm{1}}}),
\end{split}
\end{align*}
%\noindent and
\begin{align*}
\begin{split}
\iota_{X \otimes (X \otimes X)}^X \coloneqq (\id_X \otimes \iota_{X,X}^{{\mathbbm{1}}}) \circ \hphantom{\lambda}\nhphantom{$\rho$}\rho_X^{-1} = [v_{{\mathbbm{1}}}(X, X, X \otimes X)\hphantom{u}\nhphantom{$v$}](\rho_X^{-1}\hphantom{\lambda}\nhphantom{$\rho$} \otimes_\mathbbm{k} \iota_{X,X}^{{\mathbbm{1}}})\hphantom{,}
\end{split}
\end{align*}
\noindent for $\homset(X, (X \otimes X) \otimes X)$ and $\homset(X, X \otimes (X \otimes X))$, respectively. These are once again both $1$-dimensional spaces, so let's define a constant $\omega_1 \in \mathbbm{k}^\times$ for which
\begin{align*}
\begin{split}
\Phi_{X,X,X}^X(\iota_{(X \otimes X) \otimes X}^X) \eqqcolon \omega_1 \cdot \iota_{X \otimes (X \otimes X)}^X.
\end{split}
\end{align*}
\noindent To gain constraints on $\omega_1$, we must use the pentagon equation. To this end, consider the Hom-spaces
\begin{align*}
\begin{split}
\homset({\mathbbm{1}}, ({\mathbbm{1}} \otimes X) \otimes X),\\
\homset({\mathbbm{1}}, {\mathbbm{1}} \otimes (X \otimes X)),\\
\homset({\mathbbm{1}}, (X \otimes {\mathbbm{1}}) \otimes X),\\
\homset({\mathbbm{1}}, X \otimes ({\mathbbm{1}} \otimes X)),\\
\homset({\mathbbm{1}}, (X \otimes X) \otimes {\mathbbm{1}}),\\
\homset({\mathbbm{1}}, X \otimes (X \otimes {\mathbbm{1}})).
\end{split}
\end{align*}
\newpage
\noindent We can write bases for these Hom-spaces in terms of the bases we chose previously. That is,
\begin{align*}
\begin{split}
\iota_{({\mathbbm{1}} \otimes X) \otimes X}^{\mathbbm{1}} &\coloneqq (\lambda_X^{-1} \otimes \id_X) \circ \iota_{X,X}^{{\mathbbm{1}}},\\
\iota_{{\mathbbm{1}} \otimes (X \otimes X)}^{\mathbbm{1}} &\coloneqq (\id_{\mathbbm{1}} \otimes \iota_{X,X}^{{\mathbbm{1}}}) \circ \lambda_{\mathbbm{1}}^{-1},\\
\iota_{(X \otimes {\mathbbm{1}}) \otimes X}^{\mathbbm{1}} &\coloneqq (\rho_X^{-1} \otimes \id_X) \circ \iota_{X,X}^{{\mathbbm{1}}},\\
\iota_{X \otimes ({\mathbbm{1}} \otimes X)}^{\mathbbm{1}} &\coloneqq (\id_X \otimes \lambda_X^{-1}) \circ \iota_{X,X}^{{\mathbbm{1}}},\\
\iota_{(X \otimes X) \otimes {\mathbbm{1}}}^{\mathbbm{1}} &\coloneqq (\iota_{X,X}^{{\mathbbm{1}}} \otimes \id_{\mathbbm{1}}) \circ \lambda_{\mathbbm{1}}^{-1},\\
\iota_{X \otimes (X \otimes {\mathbbm{1}})}^{\mathbbm{1}} &\coloneqq (\id_X \otimes \rho_X^{-1}) \circ \iota_{X,X}^{{\mathbbm{1}}}.
\end{split}
\end{align*}
\noindent This culminates in the three new constants given by
\begin{align*}
\begin{split}
\Phi_{{\mathbbm{1}},X,X}^{\mathbbm{1}}(\iota_{({\mathbbm{1}} \otimes X) \otimes X}^{\mathbbm{1}}) &\eqqcolon \omega_2 \cdot \iota_{{\mathbbm{1}} \otimes (X \otimes X)}^{\mathbbm{1}},\\
\Phi_{X,{\mathbbm{1}},X}^{\mathbbm{1}}(\iota_{(X \otimes {\mathbbm{1}}) \otimes X}^{\mathbbm{1}}) &\eqqcolon \omega_3 \cdot \iota_{X \otimes ({\mathbbm{1}} \otimes X)}^{\mathbbm{1}},\\
\Phi_{X,X,{\mathbbm{1}}}^{\mathbbm{1}}(\iota_{(X \otimes X) \otimes {\mathbbm{1}}}^{\mathbbm{1}}) &\eqqcolon \omega_4 \cdot \iota_{X \otimes (X \otimes {\mathbbm{1}})}^{\mathbbm{1}}.
\end{split}
\end{align*}
\noindent The final bases we need are for the Hom-spaces
\begin{align*}
\begin{split}
\homset({\mathbbm{1}}, ((X \otimes X) \otimes X) \otimes X),\\
\homset({\mathbbm{1}}, (X \otimes (X \otimes X)) \otimes X),\\
\homset({\mathbbm{1}}, (X \otimes X) \otimes (X \otimes X)),\\
\homset({\mathbbm{1}}, X \otimes ((X \otimes X) \otimes X)),\\
\homset({\mathbbm{1}}, X \otimes (X \otimes (X \otimes X))).
\end{split}
\end{align*}
\noindent Following the same procedure as before, we have
\begin{align*}
\begin{split}
\iota_{((X \otimes X) \otimes X) \otimes X}^{\mathbbm{1}} &\coloneqq (\iota_{(X \otimes X) \otimes X}^X \otimes \id_X) \circ \iota_{X,X}^{{\mathbbm{1}}} = ((\iota_{X,X}^{{\mathbbm{1}}} \otimes \id_X) \otimes \id_X) \circ \iota_{({\mathbbm{1}} \otimes X) \otimes X}^{\mathbbm{1}},\\
\iota_{(X \otimes (X \otimes X)) \otimes X}^{\mathbbm{1}} &\coloneqq (\iota_{X \otimes (X \otimes X)}^X \otimes \id_X) \circ \iota_{X,X}^{{\mathbbm{1}}} = ((\id_X \otimes \iota_{X,X}^{{\mathbbm{1}}}) \otimes \id_X) \circ \iota_{(X \otimes {\mathbbm{1}}) \otimes X}^{\mathbbm{1}},\\
\iota_{(X \otimes X) \otimes (X \otimes X)}^{\mathbbm{1}} &\coloneqq (\iota_{X,X}^{{\mathbbm{1}}} \otimes (\id_X \otimes \id_X)) \circ \iota_{{\mathbbm{1}} \otimes (X \otimes X)}^{\mathbbm{1}} = ((\id_X \otimes \id_X) \otimes \iota_{X,X}^{{\mathbbm{1}}}) \circ \iota_{(X \otimes X) \otimes {\mathbbm{1}}}^{\mathbbm{1}},\\ %(\iota_{X,X}^{{\mathbbm{1}}} \otimes \iota_{X,X}^{{\mathbbm{1}}}) \circ \lambda_{\mathbbm{1}}^{-1},\\
\iota_{X \otimes ((X \otimes X) \otimes X)}^{\mathbbm{1}} &\coloneqq (\id_X \otimes \iota_{(X \otimes X) \otimes X}^X) \circ \iota_{X,X}^{{\mathbbm{1}}} = (\id_X \otimes (\iota_{X,X}^{\mathbbm{1}} \otimes \id_X)) \circ \iota_{X \otimes ({\mathbbm{1}} \otimes X)}^{\mathbbm{1}},\\
\iota_{X \otimes (X \otimes (X \otimes X))}^{\mathbbm{1}} &\coloneqq (\id_X \otimes \iota_{X \otimes (X \otimes X)}^X) \circ \iota_{X,X}^{{\mathbbm{1}}} = (\id_X \otimes (\id_X \otimes \iota_{X,X}^{{\mathbbm{1}}})) \circ \iota_{X \otimes (X \otimes {\mathbbm{1}})}^{\mathbbm{1}}.
\end{split}
\end{align*}
\noindent The pentagon diagram thus gives us
\begin{center}
\begin{tikzcd}[arrows={|->}, row sep=1.2cm, column sep=2cm]
& \iota_{((X \otimes X) \otimes X) \otimes X}^{\mathbbm{1}}\arrow[dl, "\alpha_{X,X,X} \otimes \id_X"']\arrow[dr, "\alpha_{X \otimes X,X,X}"] &\\
\omega_1 \cdot \iota_{(X \otimes (X \otimes X)) \otimes X}^{\mathbbm{1}}\arrow[d, "\alpha_{X,X \otimes X,X}"'] & & \omega_2 \cdot \iota_{(X \otimes X) \otimes (X \otimes X)}^{\mathbbm{1}}\arrow[d, "\alpha_{X,X,X \otimes X}"]\\
\omega_1\omega_3 \cdot \iota_{X \otimes ((X \otimes X) \otimes X)}^{\mathbbm{1}}\arrow[r, "\id_X\otimes\alpha_{X,X,X}"'] & \omega_1^2\omega_3 \cdot \iota_{X \otimes (X \otimes (X \otimes X))}^{\mathbbm{1}}\arrow[r, equal] & \omega_2\omega_4 \cdot \iota_{X \otimes (X \otimes (X \otimes X))}^{\mathbbm{1}}\\
\end{tikzcd}.
\end{center}
\newpage
\noindent This gives us the relation $\omega_1^2\omega_3 = \omega_2\omega_4$. However, there could be more constraints on these constants! As it turns out, though, there is a simple way of determining these. Suppose we use the more insightful labeling convention
\begin{align*}
\begin{split}
\omega(X, X, X) &\coloneqq \omega_1,\\
\omega(\nhphantom{${\mathbbm{1}}$}\hphantom{X}{\mathbbm{1}}, X, X) &\coloneqq \omega_2,\\
\omega(X, \nhphantom{${\mathbbm{1}}$}\hphantom{X}{\mathbbm{1}}, X) &\coloneqq \omega_3,\\
\omega(X, X, \nhphantom{${\mathbbm{1}}$}\hphantom{X}{\mathbbm{1}}) &\coloneqq \omega_4.
\end{split}
\end{align*}
\noindent The reason why we have chosen this relabeling is because now acting by $\alpha_{A, B, C}$ in the pentagon diagram introduces a factor of $\omega(A, B, C)$. If we apply this to the general pentagon diagram
\begin{center}
\begin{tikzcd}[row sep=1.2cm, column sep=2cm]
& ((A \otimes B) \otimes C) \otimes D\arrow[dl, "\alpha_{A,B,C} \otimes \id_D"']\arrow[dr, "\alpha_{A \otimes B,C,D}"] &\\
(A \otimes (B \otimes C)) \otimes D\arrow[d, "\alpha_{A,B \otimes C,D}"'] & & (A \otimes B) \otimes (C \otimes D)\arrow[d, "\alpha_{A,B,C\otimes D}"]\\
A \otimes ((B \otimes C) \otimes D)\arrow[rr, "\id_A\otimes\alpha_{B,C,D}"'] & & A \otimes (B \otimes (C \otimes D))
\end{tikzcd},
\end{center}
\noindent we obtain the constraints
\begin{align*}
\begin{split}
\omega(A, B, C)\omega(A, BC, D)\omega(B, C, D) = \omega(AB, C, D)\omega(A, B, CD)
\end{split}
\end{align*}
\noindent for all $A, B, C, D \in \obset(\textcat{Vec}_{\mathbb{Z}/2\mathbb{Z}})$. This is exactly the definition for a $3$-cocycle $\omega : (\mathbb{Z}/2\mathbb{Z})^{\oplus 3} \to \mathbbm{k}^\times$! In general, given a finite Abelian group $G$, there is a bijective correspondence between cohomologous $3$-cocycles on $G$ and monoidal equivalence classes of $6j$-symbols. We shall henceforth denote by $\textcat{Vec}_G^\omega$ the category of $G$-graded vector spaces whose $6j$-symbols are given by the non-trivial $3$-cocycle $\omega : G \times G \times G \to \mathbbm{k}^\times$, and $\textcat{Vec}_G$ the category whose $6j$-symbols are trivial.\\
\end{example}

\noindent Consider $\textcat{Vec}_G^\omega$, for some non-trivial $3$-cocycle $\omega$. By the strictification theorem of Mac Lane, this category is monoidally equivalent to a strict monoidal category, say $\mathcal{S}$. But if the associator for $\mathcal{S}$ is trivial, wouldn't the $6j$-symbols too be trivial? Well, suppose that for any pair $g, h \in G$, we pick an isomorphism $\mu_{g,h} : g \otimes h \to gh$. Note that in the skeletal setting $g \otimes h = gh$, but in the strict setting they are only guaranteed to be isomorphic. Because $\homset(g \otimes (h \otimes k), (g \otimes h) \otimes k)$ is $1$-dimensional,
\begin{align*}
\begin{split}
\mu_{g,hk} \circ (\id_g \otimes \mu_{h,k}) = \omega(g, h, k) \mu_{gh,k} \circ (\mu_{g,h} \otimes \id_k)
\end{split}
\end{align*}
\noindent for some $\omega(g, h, k) \in \mathbbm{k}^\times$. Doing this for each vertex of the pentagon diagram, we recover that $\omega$ must be a $3$-cocycle. If we choose another isomorphism, say $\mu'_{g,h}$, then it must satisfy $\mu'_{g,h} = \beta(g,h)\mu_{g,h}$ for some $\beta(g, h) \in \mathbbm{k}^\times$. Replacing $\mu$ with $\mu'$ gives us a new scalar
\begin{align*}
\begin{split}
\omega'(g, h, k) = \beta(h, k)^{-1}\beta(g, hk)^{-1}\beta(gh, k)\beta(g, h)\omega(g, h, k).
\end{split}
\end{align*}
\noindent In other words, it multiplies $\omega$ by a $3$-coboundary, giving us a cohomologous $3$-cocycle.
\newpage

\ruledsection{The Cuntz Algebra Approach of Izumi}{2}
%\noindent\\ Take $\textcat{Vec}_G$ to be skeletal. Consider an associativity constraint $a_{ghk} : ghk \to ghk$. Since $ghk$ is a simple object, $\homset(ghk, ghk) \cong \mathbbm{k}$, whence $a_{ghk} = \mu_{ghk}\textup{id}_{ghk}$ for some $\mu_{ghk} \in \mathbbm{k}^\times$. Note that the pentagon diagram enforces certain conditions on our choice of $\mu_{ghk}$; in particular, if we look at this diagram, we'll see that $\mu_{ghk} = \omega(g, h, k)$ for some $3$-cocycle $\omega$. By this, we mean a map $\omega : G \times G \times G \to \mathbbm{k}^\times$ satisfying
%\begin{align*}
%\begin{split}
%\omega(x, y, zw)\omega(xy, z, w)\omega(y, z, w)^{-1}\omega(x, yz, w)^{-1}\omega(x, y, z)^{-1} = 1
%\end{split}
%\end{align*}
%\noindent for all $x, y, z, w \in G$. We will henceforth denote by $\textcat{Vec}_G^\omega$ the category of $G$-graded vector spaces with associativity constraint $a_{ghk} = \omega(g, h, k)\textup{id}_{ghk}$, for all $g, h, k \in G$, and $\textcat{Vec}_G$ the category of $G$-graded vector spaces with trivial associativity.\\

\noindent Consider the category $\textup{End}(M)$, for $M$ a hyperfinite type III factor. This category is strict, as $\rho \otimes \sigma \coloneqq \rho \circ \sigma$ by definition. Every near-group category with group $G$ contains some copy of $\textcat{Vec}_G^\omega$ corresponding to the group-like part. Because every unitary near-group category is a subcategory of $\textup{End}(M)$ and is hence itself strict, we know that it will actually contain the ``strictification'' of some $\textcat{Vec}_G^\omega$. However, Izumi shows that if $\mathcal{C}$ is any fusion category containing a simple object that is fixed under tensor products with invertibles (that is, there exists some simple object $X$ such that $X \otimes g \cong X$ for all invertible $g$), then it contains a copy of $\textcat{Vec}_G$, for $G$ the group of isomorphism classes of invertible objects. He shows in addition that if the fusion category is also unitary, then $g \otimes X = X$ (but we may not necessarily have that $X \otimes g = X$). The upshot is that we almost know how objects are tensored, since the group-like part will have trivial associativity (that is, $g \otimes h = gh$). We just need to understand $X \otimes g$ and $X \otimes X$, as well as the morphisms.\\

\noindent In \cite{Izu17}, Izumi showed that every unitary near-group category $\mathcal{C}$ with multiplicity $m$ is equivalent to a subcategory of $\textup{End}(M)$, where $M$ is the hyperfinite type $\textup{III}_1$ factor. In particular, it is generated by a single irreducible endomorphism $\rho \in \textup{End}_0(M)$ satisfying the fusion rules
\begin{align*}
\begin{split}
[\alpha_g] \otimes [\alpha_h] = [\alpha_{gh}],\\
[\alpha_g] \otimes [\rho] = [\rho] \otimes [\alpha_g] = [\rho],\\
[\rho] \otimes [\rho] = \bigoplus_{g \in G} [\alpha_g] \oplus [\rho]^{\oplus m},\\
\end{split}
\end{align*}
\noindent where the map $\alpha : G \to \textup{Aut}(M)$ induces an injective homomorphism from $G$ into $\textup{Out}(M)$.\\

\noindent The main result of \cite{Izu17} is \cite[Theorem 4.9]{Izu17}. Essentially, there is a bijective correspondence between the set of equivalence classes of unitary near-group categories with finite group $G$ and multiplicity parameter $m$ and the set of equivalence classes of admissible tuples $(\mathcal{K}, j_1, j_2, V, U_\mathcal{K}, \chi, l)$ (see \cite[Definition 4.8]{Izu17}). Here $\mathcal{K}$ is the finite-dimensional Hilbert space $\homset(\rho, \rho^2)$, $j_1$ and $j_2$ are two antilinear isometries of $\mathcal{K}$, $V$ and $U_\mathcal{K}$ are unitary representations of $G$ on $\mathcal{K}$, $\{\chi_g\}_{g \in G}$ are characters of $G$ and $l$ is a linear map from $\mathcal{K}$ to the set $\mathcal{B}(\mathcal{K}, \mathcal{K} \otimes \mathcal{K})$ of bounded operators $\mathcal{K} \to \mathcal{K} \otimes \mathcal{K}$.\\

\noindent By \cite[Theorem 9.1]{Izu17}, the unitary near-group categories with finite Abelian group $G$ and $m = \abs{G}$ are completely classified tuples of the form $(\langle\cdot,\cdot\rangle, a, b, c)$, where $\langle\cdot,\cdot\rangle : G \times G \to \mathbb{T}$ is a non-degenerate symmetric bicharacter and where $a : G \to \mathbb{T}$, $b : G \to \mathbb{T}$ and $c \in \mathbb{T}$ satisfy various conditions. When we say that $\langle\cdot,\cdot\rangle$ is a bicharacter, we mean that
\begin{align*}
\begin{split}
\langle xy, z\rangle = \langle x, z\rangle\langle y, z\rangle\qquad\text{and}\qquad\langle x, yz\rangle = \langle x, y\rangle\langle x, z\rangle
\end{split}
\end{align*}
for all $x, y, z \in G$. By non-degenerate, we mean that
\begin{align*}
\begin{split}
\langle x, \cdot\rangle = \langle y, \cdot\rangle
\end{split}
\end{align*}
\noindent if and only if $x = y$. This is equivalent to the map $\varphi : G \to \homset(G, \mathbb{T})$ given by $x \mapsto \langle x, \cdot\rangle$ being an isomorphism.\\

\noindent\begin{definition}\textup{(Cuntz Algebra).} Let $\{S_i\}_{i=1}^n$ be a set of isometries on an infinite-dimensional Hilbert space $\mathscr{H}$. Suppose moreover that these isometries satisfy the {\em Cuntz relation}
\begin{align*}
\begin{split}
\sum_{k=1}^n S_kS_k^* = 1.
\end{split}
\end{align*}
\noindent The {\em Cuntz algebra $\mathcal{O}_n$} is the universal $C^*$-algebra $C^*(S_1, \dots, S_n)$.\\
\end{definition}

\noindent\begin{remark} Note that, as isometries, $S_i^*S_i = 1$. In particular, we must have that $S_i^*S_j = \delta_{i,j}$ for all $i, j \in \{1, \dots, n\}$. This follows from the fact that a sum of projections is itself a projection if and only if the projections in the sum are pairwise orthogonal. The Cuntz relation is essentially ensuring that the sum of the projections $S_iS_i^*$ is the trivial projection.\\
\end{remark}

\noindent 
\newpage

\noindent\begin{example}\textup{(Fibonacci Category).} Let's look at the Fibonacci category. This is the near-group with $G = \{0\}$ and $m = 1$. Our choice for $\langle\cdot,\cdot\rangle$ is obvious, and \cite[Lemma 7.1]{Izu17} tells us that
\begin{align*}
\begin{split}
c^3 a(0) = \csqrt{n} = 1 \implies a(0) = c^{-3}.
\end{split}
\end{align*}
\noindent Moreover, \cite[Theorem 9.1]{Izu17} tells us that $b$ is defined by $b : 0 \mapsto -1/d$, where $d$ corresponds to the dimension of our irreducible generator $\rho$. Let's determine $c$ and $d$. Because $b$ is equal to its own Fourier transform, \cite[Theorem 9.1]{Izu17} tells us that
\begin{align*}
\begin{split}
b(0) = ca(0)b(0) \implies a(0) = c^{-1}.
\end{split}
\end{align*}
\noindent In order for $c^{-1} = c^{-3}$, we require $c = \pm 1$. Finally, \cite[Equation 9.5]{Izu17} tells us that
\begin{align*}
\begin{split}
b(0)b(0)b(0) &= b(0)b(0) \mp \frac{1}{d},
\end{split}
\end{align*}
\begin{align*}
\begin{split}
\implies &-\frac{1}{d^3} = \frac{1}{d^2} \mp \frac{1}{d},\\
\implies &\pm d^2 - d - 1 = 0.
\end{split}
\end{align*}
\noindent This only has a real solution when $c = 1$, whence $d$ is nothing but the golden ratio (as it cannot be negative in the unitary case - this alternative solution, known as the Galois dual, corresponds to a non-unitary near-group in this case and many others). This is exactly what we would expect, as $d$ is the dimension of $X$ (where $d^2 = 1 + d$ comes from the fusion rule $X^2 = \mathbbm{1} \oplus X$).\\
\end{example}

\noindent\begin{example}\textup{($G = \mathbb{Z}/2\mathbb{Z}$).} Let's look at the case where $G = \mathbb{Z}/2\mathbb{Z}$ and $m = 2$. This near-group corresponds to the even part of the type $A_4$ subfactor. We know the dimension is
\begin{align*}
\begin{split}
d_\pm \coloneqq \frac{m \pm \csqrt{m^2 + 4n}}{2} = 1 \pm \csqrt{3}.
\end{split}
\end{align*}
\noindent In the unitary setting, we of course ask that $d$ be positive, and hence we choose $d = d_+$. The only possibility for a non-degenerate bicharacter is
\begin{align*}
\begin{split}
\langle 0, 0\rangle = 1,\qquad\langle 0, 1\rangle = \langle 1, 0\rangle = 1\qquad\text{and}\qquad\langle 1, 1\rangle = -1.
\end{split}
\end{align*}
\noindent From \cite[Equation 7.8]{Izu17}, it follows that
\begin{align*}
\begin{split}
a(0) = 1\qquad\text{and}\qquad a(1) = \pm i.
\end{split}
\end{align*}
\noindent Meanwhile, \cite[Equation 9.4]{Izu17} tells us that
\begin{align*}
\begin{split}
\overline{b(1)} = \pm i b(1) \implies \mathfrak{R}(b(1)) = \mp \mathfrak{I}(b(1)),
\end{split}
\end{align*}
\noindent whence \cite[Equation 9.3]{Izu17} gives us
\begin{align*}
\begin{split}
\mathfrak{R}(b(1))^2 + \mathfrak{I}(b(1))^2 = (b(1)\overline{b(1)})^2 = \frac{1}{2} \implies b(1) = \frac{1 - a(1)}{2}.
\end{split}
\end{align*}
\noindent It then follows from evaluating \cite[Equation 9.1]{Izu17} with $g = 0$ and rearranging for $c$ that
\begin{align*}
\begin{split}
c = \frac{1 - \csqrt{3} + a(1)(1 + \csqrt{3})}{2\csqrt{2}}.
\end{split}
\end{align*}
\noindent Note that we may choose either $a(1) = i$ or $a(1) = -i$; both of these lead to solutions. Moreover, in the non-unitary setting, we may take the Galois conjugate of $d$.
\end{example}
\newpage

\noindent\begin{example}\label{UHIZ2}\textup{($G = \mathbb{Z}/2\mathbb{Z}$).} Let's determine the Haagerup--Izumi categories with $G = \mathbb{Z}/2\mathbb{Z}$. Let
\begin{align*}
\begin{split}
d_\pm \coloneqq \frac{n \pm \csqrt{n^2 + 4}}{2},
\end{split}
\end{align*}
\noindent where in this example $d \coloneqq 1 + \csqrt{2}$. Izumi's classification involves a triplet $(\epsilon_h(g), \omega(g), A_{h,k}(g))$, where $\epsilon_h(g) \in \{-1, 1\}$, $\omega(g) \in \mathbb{T}$ and $A_{h,k}(g) \in \mathbb{C}$ satisfy \cite[Equations 4.1--4.9]{Izu18}. Well, we know
\begin{align*}
\begin{split}
\epsilon_0(0) = \epsilon_1(0) = 1\qquad\text{and}\qquad\epsilon_0(1) = \epsilon_0(1)\epsilon_0(1) \implies \epsilon_0(1) = 1.
\end{split}
\end{align*}
\noindent By \cite[Equation 4.7]{Izu18},
\begin{align*}
\begin{split}
A_{0,0}(g) = A_{0,0}(g)\omega(g),
\end{split}
\end{align*}
\noindent which tells us that either $\omega(g) = 1$ or $A_{0,0}(g) = 0$ for each $g \in G$. Let's fix any $g \in G$ and consider the case when $A_{0,0}(g) = 0$. In this case, however, \cite[Equations 4.3 and 4.4]{Izu18} give us
\begin{align*}
\begin{split}
A_{1,0}(g)\overline{A_{\delta_{g,0}-g,0}(g)} = 1 - \frac{\abs{\omega(g)}}{d} \implies \text{``}\frac{1}{d^2} = 1 - \frac{1}{d}\text{''}.
\end{split}
\end{align*}
\noindent This ``equality'' is nonsense; we must therefore have $\omega(g) = 1$ for all $g \in G$. Suppose now that $\epsilon_1(1) = 1$. Then \cite[Equation 4.7]{Izu18} gives us
\begin{align*}
\begin{split}
A_{0,1}(0) = A_{1,1}(0) = A_{1,0}(0)\qquad\text{and}\qquad A_{0,1}(1) = A_{1,1}(1) = A_{1,0}(1),
\end{split}
\end{align*}
\noindent while \cite[Equation 4.8]{Izu18} gives us $A_{1,1}(0) = A_{1,1}(1)$. Now, \cite[Equations 4.4 and 4.6]{Izu18} tell us
\begin{align*}
\begin{split}
A_{0,1}(0) A_{1,1}(1) + A_{1,1}(0) A_{1,0}(1) = 0.
\end{split}
\end{align*}
\noindent Thus $A_{0,1}(g) = A_{1,1}(g) = A_{1,0}(g) = 0$ and hence $A_{0,0}(g) = -1/d$ by \cite[Equation 4.3]{Izu18}. However, in this case we cannot satisfy \cite[Equation 4.9]{Izu18}. Suppose instead that $\epsilon_1(1) = -1$. With this new $2$-cocycle, \cite[Equation 4.7]{Izu18} now gives us
\begin{align*}
\begin{split}
A_{0,1}(0) = A_{1,1}(0) = A_{1,0}(0)\qquad\text{and}\qquad A_{0,1}(1) = -A_{1,1}(1) = A_{1,0}(1),
\end{split}
\end{align*}
\noindent while \cite[Equation 4.8]{Izu18} gives us $A_{1,1}(1) = -A_{1,1}(0)$. We then see by \cite[Equation 4.4]{Izu18} that
\begin{align*}
\begin{split}
A_{0,1}(0) A_{1,0}(0) + A_{1,1}(0) A_{1,1}(0) = 1 \implies A_{1,0}(0) = \pm\frac{1}{\csqrt{2}} = \pm\frac{1}{d-1},
\end{split}
\end{align*}
\noindent and by \cite[Equation 4.9]{Izu18} that
\begin{align*}
\begin{split}
A_{0,0}(0)A_{1,0}(0)^2 = A_{1,0}(0)^2 + A_{1,0}(0)^3 \implies A_{0,0}(0) = 1 + A_{1,0}(0) = \frac{d-1 \pm 1}{d-1}.
\end{split}
\end{align*}
\noindent Finally, \cite[Equation 4.3]{Izu18} allows us to deduce
\begin{align*}
\begin{split}
A_{1,0}(0) = -\frac{1}{d-1},
\end{split}
\end{align*}
\noindent whence
\begin{align*}
\begin{split}
A(0) = \frac{1}{d-1}\begin{pmatrix}d-2 & -1\\ -1 & -1\end{pmatrix}\qquad\text{and}\qquad A(1) = \frac{1}{d-1}\begin{pmatrix}d-2 & -1\\ -1 & \hphantom{-}1\end{pmatrix}.
\end{split}
\end{align*}
\noindent This category is nothing but the even part of the type $A_7$ subfactor.\\
\end{example}
\newpage

\noindent\begin{remark} Suppose that $\abs{G}$ is odd. Then \cite[Equation 4.1]{Izu18} tells us that $\epsilon_h(g) = 1$, while \cite[Equation 4.2]{Izu18} tells us that $\omega(g)$ does not depend on $g$. Moreover, $A_{h,k}(g)$ cannot depend on $g$ by \cite[Equation 4.5]{Izu18}, and either $\omega = 1$ or $A_{0,0} = 0$ by \cite[Equation 4.7]{Izu18}. In this case, \cite[Equations 4.1--4.9]{Izu18} reduce to the following four equations.
\begin{align*}
\begin{split}
A_{h,k} = A_{-k,h-k}\omega = A_{k-h,-h}\overline{\omega},
\end{split}
\end{align*}
\begin{align*}
\begin{split}
\sum_{h \in G} A_{h,0} = -\frac{\overline{\omega}}{d_\pm},
\end{split}
\end{align*}
\begin{align*}
\begin{split}
\sum_{h \in G} A_{h-g,k} A_{k,h-g'} = \delta_{g,g'} - \frac{\delta_{k,0}}{d_\pm},
\end{split}
\end{align*}
\begin{align*}
\begin{split}
\sum_{l \in G} A_{x+y,l} A_{-x,l+p} A_{-y,l+q} = A_{p+x,q+x+y} A_{q+y,p+x+y} - \frac{\delta_{x,0}\delta_{y,0}}{d_\pm}.
\end{split}
\end{align*}
\end{remark}
\noindent The first three equations above are precisely \cite[Equations 4.7, 4.8 and 4.9]{EG17}! In particular, to see that our third equation is equivalent to \cite[Equation 4.9]{EG17}, we simply make the change of variables $\hat{g} \coloneqq g' - g$ and $\hat{h} \coloneqq h - g'$, whence we obtain
\begin{align*}
\begin{split}
\sum_{\hat{h} \in G} A_{\hat{h}+\hat{g},k} A_{k,\hat{h}} = \delta_{\hat{g},0} - \frac{\delta_{k,0}}{d_\pm}.
\end{split}
\end{align*}
\noindent Similarly, using our first equation while making the change of variables $\hat{l} \coloneqq l-x-y$, $\hat{p} \coloneqq p+x+y$, $\hat{q} \coloneqq q+x+y$, $\hat{x} \coloneqq -x$ and $\hat{y} \coloneqq -y$, our fourth equation becomes
\begin{align*}
\begin{split}
\overline{\omega}\sum_{\hat{l} \in G} A_{\hat{l},\hat{x}+\hat{y}} A_{\hat{x},\hat{l}+\hat{p}} A_{\hat{y},\hat{l}+\hat{q}} = A_{\hat{y}+\hat{p},\hat{q}} A_{\hat{x}+\hat{q},\hat{p}} - \frac{\delta_{\hat{x},0}\delta_{\hat{y},0}}{d_\pm},
\end{split}
\end{align*}
\noindent showing that it is equivalent to \cite[Equation 4.11]{EG17}.
\newpage

\ruledsection{The Leavitt Algebra Approach of Evans--Gannon}{3}
\noindent\\ The important results are \cite[Theorem 1 and Theorem 2]{EG17}.\\

\noindent\begin{definition}\textup{(Leavitt Algebra).} Let $X \coloneqq (x_{ij})$ and $Y \coloneqq (y_{ij})$ be $m \times n$ and $n \times m$ matrices of symbols, respectively. The {\em Leavitt $K$-algebra of type $(m, n)$} is the free associative unital $K$-algebra
\begin{align*}
\begin{split}
\mathcal{L}_K(m, n) \coloneqq \frac{K[x_{ij}, y_{ij}]}{\langle XY = I_m, YX = I_n\rangle}.
\end{split}
\end{align*}
\noindent In other words, it is the universal $K$-algebra with generators
\begin{align*}
\begin{split}
\{x_{ij} : 1 \leq i \leq m, 1 \leq j \leq n\} \sqcup \{y_{ij} : 1 \leq i \leq n, 1 \leq j \leq m\}
\end{split}
\end{align*}
\noindent and {\em Leavitt--Cuntz relations}
\begin{align*}
\begin{split}
\sum_{k=1}^m y_{ik} x_{kj} = \delta_{i,j}\qquad\text{and}\qquad\sum_{k=1}^n x_{ik} y_{kj} = \delta_{i,j},
\end{split}
\end{align*}
\noindent for all suitable $i, j$.\\
\end{definition}

\noindent Consider the Leavitt $\mathbb{C}$-algebra of type $(1, n)$, which we shall henceforth denote by $\mathcal{L}_n \coloneqq \mathcal{L}_\mathbb{C}(1, n)$. We have that $\mathcal{O}_n = C^*(\mathcal{L}_n)$, where $x_i = S_i$ and $y_i = S_i^*$. Let's think about what this means precisely. The Leavitt--Cuntz relations for $m = 1$ become
\begin{align*}
\begin{split}
y_i x_j = \delta_{i,j}\qquad\text{and}\qquad\sum_{k=1}^n x_k y_k = 1.
\end{split}
\end{align*}
\noindent We may endow $\mathcal{L}_n$ with the structure of a $*$-algebra by defining a conjugate homogeneous\linebreak antihomomorphism that sends $x_i \mapsto y_i$ and $y_i \mapsto x_i$. We further define
\begin{align*}
\begin{split}
\norm{a} \coloneqq \sup\{p(a) : \textup{$p$ is a $C^*$-seminorm on $\mathcal{L}_n$}\}
\end{split}
\end{align*}
\noindent for all $a \in \mathcal{L}_n$, where a $C^*$-seminorm is just a seminorm for which $p(a^*a) = p(a)^2$ and $p(ab) \leq p(a)p(b)$. Note that $0 \leq \norm{a} \leq 1$, as $1 = \norm{y_ix_i} = \norm{x_i^2}$ and hence $\norm{x_i} = \norm{y_i} = 1$ for all $i$. The condition $p(ab) \leq p(a)p(b)$ ensures that $\mathcal{I} \coloneqq \{a \in \mathcal{L}_n : \norm{a} = 0\}$ is an ideal in $\mathcal{L}_n$. Our $C^*$-seminorm then descends to a $C^*$-norm on the quotient $\mathcal{L}_n/\mathcal{I}$. The completion of $\mathcal{L}_n/\mathcal{I}$ with respect to this $C^*$-norm is known as the universal $C^*$-algebra of $\mathcal{L}_n$, denoted by $C^*(\mathcal{L}_n)$. This is precisely $\mathcal{O}_n$ by definition. We may therefore view $\mathcal{L}_n$ as the polynomial part of $\mathcal{O}_n$.\\

\noindent Because $\mathcal{L}_n$ is a unital algebra over $\mathbb{C}$, its algebra endomorphisms define a strict preadditive $\mathbb{C}$-linear tensor category $\mathcal{E}(\mathcal{L}_n)$. The objects of $\mathcal{E}(\mathcal{L}_n)$ are algebra endomorphisms of $\mathcal{L}_n$, while the morphisms are intertwiners. In other words, a morphism $r \in \homset_{\mathcal{E}(\mathcal{L}_n)}(\beta, \gamma)$ is an element $r \in \mathcal{L}_n$ for which $r\beta(x) = \gamma(x)r$ for all $x \in \mathcal{L}_n$, with composition being multiplication in $\mathcal{L}_n$. We endow $\mathcal{E}(\mathcal{L}_n)$ with a tensor product given on objects as $\beta \otimes \gamma \coloneqq \beta \circ \gamma$ for all $\beta, \gamma \in \obset(\mathcal{E}(\mathcal{L}_n))$ and on morphisms as $r \otimes s \coloneqq r\beta(s) = \gamma(s)r \in \homset_{\mathcal{E}(\mathcal{L}_n)}(\beta\circ\rho, \gamma\circ\sigma)$ for all $r \in \homset_{\mathcal{E}(\mathcal{L}_n)}(\beta, \gamma)$ and $s \in \homset_{\mathcal{E}(\mathcal{L}_n)}(\rho, \sigma)$. Note that this is indeed well-defined, since
\begin{align*}
\begin{split}
r\beta(s)\beta(\rho(x)) = \gamma(s)r\beta(\rho(x)) = \gamma(s)\gamma(\rho(x))r = \gamma(s\rho(x))r = \gamma(\sigma(x)s)r = \gamma(\sigma(x))\gamma(s)r
\end{split}
\end{align*}
\noindent for all $x \in \mathcal{L}_n$.

\noindent Let $\mathcal{A}$ now be a collection of algebra endomorphisms of $\mathcal{L}_n$ that is closed under composition and that contains the identity. We denote by $\mathcal{E}(\mathcal{A})$ the subcategory of $\mathcal{E}(\mathcal{L}_n)$ restricted to $\mathcal{A}$. This subcategory is itself a $\mathbb{C}$-linear tensor category, and $\End_{\mathcal{E}(\mathcal{A})}(\id) = \mathbb{C}$. Suppose we now take its Karoubi envelope, $\overline{\mathcal{E}(\mathcal{A})}$. Its objects consist of pairs $(p, \beta)$, where $\beta \in \mathcal{A}$ and $p \in \End_{\mathcal{E}(\mathcal{A})}(\beta)$ is an idempotent. The morphism spaces are of the form $\homset_{\overline{\mathcal{E}(\mathcal{A})}}((p, \beta), (q, \gamma)) \coloneqq q\homset_{\mathcal{E}(\mathcal{A})}(\beta, \gamma)p$, with composition given once more by multiplication. We endow $\overline{\mathcal{E}(\mathcal{A})}$ with the structure of a tensor category by defining a monoidal product on objects by $(p, \beta) \otimes (q, \gamma) \coloneqq (p \otimes q, \beta \otimes \gamma) = (p\beta(q), \beta\circ\gamma)$ and on morphisms by $(qrp) \otimes (q'r'p') \coloneqq qrp\beta(q'r'p')$, for $qrp \in \homset((p,\beta), (q,\gamma))$ and $q'r'p' \in \homset((p',\beta'), (q',\gamma'))$. We also endow $\overline{\mathcal{E}(\mathcal{A})}$ with the structure of an additive category as follows. The objects in this category are ordered $n$-tuples $((p_1, \beta_1), \dots, (p_n, \beta_n)) \eqcolon (p_1, \beta_1) \oplus \cdots \oplus (p_n, \beta_n)$, while the morphism spaces are
\begin{align*}
\begin{split}
\homset(((p_1, \beta_1), \dots, (p_n, \beta_n)), ((q_1, \gamma_1), \dots, (q_m, \gamma_m)))
\end{split}
\end{align*}
\begin{align*}
\begin{split}
\qquad\quad= \begin{pmatrix} q_1\homset(\beta_1, \gamma_1)p_1 & \cdots & q_1\homset(\beta_n, \gamma_1)p_n\\ \vdots & \ddots & \vdots\\ q_m\homset(\beta_1, \gamma_m)p_1 & \cdots & q_m\homset(\beta_n, \gamma_m)p_n\end{pmatrix}.
\end{split}
\end{align*}
\noindent Composition of morphisms in this category corresponds to matrix multiplication. The monoidal product $((p_1, \beta_1), \dots, (p_n, \beta_n)) \otimes ((q_1, \gamma_1), \dots, (q_m, \gamma_m))$ is the direct sum of each $(p_i, \beta_i) \otimes (q_j, \gamma_j)$. On morphisms, the monoidal product acts as the Kronecker product, with $(ij, i'j')$th entry given by $q_ir_{ji}p_j \otimes q_{i'}'r_{j'i'}'p_{j'}'$. We will write $\overline{\mathcal{E}(\mathcal{A})}_\oplus$ for the Karoubi envelope $\overline{\mathcal{E}(\mathcal{A})}$ extended by direct sums.\\

\noindent Suppose as before that $\mathcal{A}$ is a collection of $\mathcal{L}_n$-endomorphisms that is closed under composition and contains the identity. Suppose that $\homset(\beta, \gamma) = \homset(\widetilde{\beta}, \widetilde{\gamma})$ in $\mathcal{L}_n$ for all $\beta, \gamma \in \mathcal{A}$, where $\widetilde{\beta}(x) \coloneqq \beta(x')'$, and moreover that these Hom-spaces are all finite-dimensional. Then $\overline{\mathcal{E}(\mathcal{A})}_\oplus$ is a strict rigid semisimple $\mathbb{C}$-linear tensor category with finite-dimensional Hom-spaces by \cite[Lemma 2]{EG17}.
\newpage

\begin{example}\textup{(Yang--Lee Category).} Let $G = \{0\}$. Then \cite[Equation 4.7]{EG17} demands that
\begin{align*}
\begin{split}
A_{0,0} = \omega A_{0,0} = \overline{\omega}A_{0,0} \implies \omega = 1,
\end{split}
\end{align*}
whence \cite[Equation 4.8]{EG17} tells us that
\begin{align*}
\begin{split}
A_{0,0} = -\frac{1}{d_\pm}.
\end{split}
\end{align*}
\noindent The rest of \cite[Equations 4.7--4.10]{EG17} are satisfied by these choices. Hence by \cite[Theorem 2]{EG17}, we have two fusion categories for $G = \{0\}$; a unitary one with $\pm = +$ (the Fibonacci category) and a non-unitary one with $\pm = -$ (the Yang--Lee category).\\
\end{example}

\noindent\begin{example}\textup{($G = \mathbb{Z}/2\mathbb{Z}$).} The equations we must satisfy for $\abs{G}$ even are given in \cite{Izu18} \textcolor{red}{(is this true?)}. Adapting our argument from \hyperref[UHIZ2]{Example \ref*{UHIZ2}}, we see that there is exactly one non-unitary Haagerup--Izumi category with $G = \mathbb{Z}/2\mathbb{Z}$. This corresponds to $\epsilon_h(g) = (-1)^{gh}$, $\omega(g) = 1$,
\begin{align*}
\begin{split}
A(0) = \frac{1}{d-1}\begin{pmatrix}d & 1\\ 1 & 1\end{pmatrix}\qquad\text{and}\qquad A(1) = \frac{1}{d-1}\begin{pmatrix}d & \hphantom{-}1\\ 1 & -1\end{pmatrix}.
\end{split}
\end{align*}
\end{example}
\newpage

\ruledsection{Realizing Non-Unitary Near-Groups}{3}
%\noindent\\ .\\

\noindent\begin{proposition}\label{CommutingInvertibles} Let $\mathcal{C}$ be a $\mathbbm{k}$-linear fusion category with group $G$ of invertible objects, and suppose there exists a simple object $X \in \obset(\mathcal{C})$ such that $g \otimes X \cong X$ for all $g \in G$. Then the full Abelian monoidal subcategory generated by $G$ is $\textcat{Vec}_G$ with trivial associativity.\\
\end{proposition}

\noindent\begin{proof} Let $\mathcal{D}$ be the subcategory generated by $G$. This category is easily seen to be fusion, and is hence equivalent to $\textcat{Vec}_G$ on the level of fusion rings. It thus suffices to show that the associativity is trivial. We first choose isomorphisms $f_g \in \homset_\mathcal{C}(g \otimes X, X)$ and $\mu_{g,h}' \in \homset_\mathcal{C}(g \otimes h, gh)$ for all $g, h \in G$. Because $\homset_\mathcal{C}((g \otimes h) \otimes X, gh \otimes X)$ is $1$-dimensional, there exists some $z \in \mathbbm{k}^\times$ for which
\begin{align*}
\begin{split}
z\mu_{g,h}' \otimes \id_X = z(\mu_{g,h}' \otimes \id_X) = f_{gh}^{-1} \circ f_g \circ (\id_g \otimes f_h) \circ \alpha_{g,h,X}.
\end{split}
\end{align*}
\noindent That is, for each pair $g, h \in G$, there exists a unique isomorphism $\mu_{g,h} \coloneqq z\mu_{g,h}'$ for which the diagram
\begin{center}
\begin{tikzcd}[/tikz/column 1/.style={column sep = 1.5cm}]
(g \otimes h) \otimes X \arrow[dd, "\alpha_{g,h,X}"'] \arrow[r, "\mu_{g,h} \otimes \id_X"] & gh \otimes X \arrow[dr, "f_{gh}"] \\
& & X\\
g \otimes (h \otimes X) \arrow[r, "\id_g \otimes f_h"'] & g \otimes X \arrow[ur, "f_g"']
\end{tikzcd}
\end{center}
%\begin{center}
%\begin{tikzcd}
% & (g \otimes h) \otimes X\arrow[ld, "\alpha_{g,h,X}"']\arrow[rd, "\mu_{g,h} \otimes \id_X"] &\\
%g \otimes (h \otimes X)\arrow[d, "\id_g \otimes f_{h}"'] & & gh \otimes X\arrow[d, "f_{gh}"]\\
%g \otimes X\arrow[rr, "f_g"'] & & X
%\end{tikzcd}
%\end{center}
\noindent commutes. From this, we obtain the commutative diagram
\begin{center}
\begin{tikzcd}[/tikz/column 1/.style={column sep = -4.0em}, /tikz/column 1/.style={column sep = -4.0em}, /tikz/column 2/.style={column sep = 2.0em}, /tikz/column 3/.style={column sep = 2.0em}, /tikz/column 4/.style={column sep = -1.0em}, /tikz/column 5/.style={column sep = -1.0em}, row sep = 2.0em]
{((g \otimes h) \otimes k) \otimes X} \arrow[dr, "{\alpha_{g \otimes h,k,X}}"] \arrow[rrrr, "{(\mu_{g,h} \otimes \id_k) \otimes \id_X}"] \arrow[dddddd, "{\alpha_{g,h,k} \otimes \id_X}"'] & & & & {(gh \otimes k) \otimes X} \arrow[dl, "{\alpha_{gh,k,X}}"'] \arrow[dddr, "{\mu_{gh,k} \otimes \id_X}"] \\
& {(g \otimes h) \otimes (k \otimes X)} \arrow[dd, "{\alpha_{g,h,k \otimes X}}"] \arrow[dr, "{\id_{g \otimes h} \otimes f_k}"] \arrow[rr, "{\mu_{g,h} \otimes \id_{k \otimes X}}"] & & {gh \otimes (k \otimes X)} \arrow[d, "{\id_{gh} \otimes f_k}"'] \\
& & {(g \otimes h) \otimes X} \arrow[dd, "{\alpha_{g,h,X}}"] \arrow[r, "{\mu_{g,h} \otimes \id_X}"'] & {gh \otimes X} \arrow[dr, "{f_{gh}}"'] \\
& {g \otimes (h \otimes (k \otimes X))} \arrow[dr, "{\id_g \otimes (\id_h \otimes f_k)}"] & & & {X} & {ghk \otimes X} \arrow[l, "{f_{ghk}}"'] \\
& & {g \otimes (h \otimes X)} \arrow[r, "{\id_g \otimes f_h}"] & {g \otimes X} \arrow[ur, "{f_g}"] \\
& {g \otimes ((h \otimes k) \otimes X)} \arrow[uu, "{\id_g \otimes \alpha_{h,k,X}}"'] \arrow[rr, "{\id_g \otimes (\mu_{h,k} \otimes \id_X)}"'] & & {g \otimes (hk \otimes X)} \arrow[u, "{\id_g \otimes f_{hk}}"] \\
{(g \otimes (h \otimes k)) \otimes X} \arrow[rrrr, "{(\id_g \otimes \mu_{h,k}) \otimes \id_X}"'] \arrow[ur, "{\alpha_{g,h \otimes k,X}}"'] & & & & {(g \otimes hk) \otimes X} \arrow[uuur, "{\mu_{g,hk} \otimes \id_X}"'] \arrow[ul, "{\alpha_{g,hk,X}}"]
\end{tikzcd}.
\end{center}
%\begin{center}
%\begin{tikzcd}[/tikz/column 1/.style={column sep = -1.75cm}, /tikz/column 2/.style={column sep = -1.75cm}, /tikz/column 3/.style={column sep = -2cm}, /tikz/column 4/.style={column sep = -0.75cm}, /tikz/column 5/.style={column sep = -0.3cm}, /tikz/column 6/.style={column sep = -1cm}, /tikz/column 7/.style={column sep = 1cm}, row sep = 0.5cm]
% & & & & ((g \otimes h) \otimes k) \otimes X\arrow[dllll, "\alpha_{g,h,k} \otimes \id_X"']\arrow[dd, "\alpha_{g \otimes h, k, X}"]\arrow[drrr, "(\mu_{g,h} \otimes \id_k) \otimes \id_X)"] & & &\\
% (g \otimes (h \otimes k)) \otimes X\arrow[dr, "\alpha_{g, h\otimes k, X}"]\arrow[dddddd, "(\id_g \otimes \mu_{h,k}) \otimes \id_X"'] & & & & & & & (gh \otimes k) \otimes X\arrow[ddl, "\alpha_{gh,k,X}"']\arrow[dddddd, "\mu_{gh,k} \otimes \id_X"]\\
% & g \otimes ((h \otimes k) \otimes X)\arrow[dr, "\id_g\otimes\alpha_{h,k,X}"]\arrow[dddd, "\id_g\otimes(\mu_{h,k} \otimes \id_X)"  {yshift=-1.125cm}] & & & (g \otimes h) \otimes (k \otimes X)\arrow[dll, "\alpha_{g,h,k\otimes X}"]\arrow[d, "\id_{g\otimes h} \otimes f_k"]\arrow[drr, "\mu_{g,h} \otimes \id_{k\otimes X}"] & & &\\
% & & g \otimes (h \otimes (k \otimes X))\arrow[dr, "\id_g \otimes (\id_h \otimes f_k)"'] & & (g \otimes h) \otimes X\arrow[dl, "\alpha_{g,h,X}"']\arrow[dr, "\mu_{g,h} \otimes \id_X"] & & gh \otimes (k \otimes X)\arrow[dl, "\id_{gh} \otimes f_k"] &\\
% & & & g \otimes (h \otimes X)\arrow[d, "\id_g \otimes f_{h}"'] & & gh \otimes X\arrow[d, "f_{gh}"] & &\\
% & & & g \otimes X\arrow[rr, "f_g"'] & & X & &\\
% & g \otimes (hk \otimes X)\arrow[urr, "\alpha_{g,hk,X}"'] & & & & & &\\
% (g \otimes hk) \otimes X\arrow[ur]\arrow[rrrrrrr, "\mu_{g,hk} \otimes \id_X"'] & & & & & & & ghk \otimes X\arrow[uull, "f_{ghk}"']
%\end{tikzcd}.
%\end{center}
%\begin{center}
%\begin{tikzcd}[/tikz/column 1/.style={column sep = -1.75cm}, /tikz/column 2/.style={column sep = -1.75cm}, /tikz/column 3/.style={column sep = -2cm}, /tikz/column 4/.style={column sep = -0.75cm}, /tikz/column 5/.style={column sep = -0.5cm}, /tikz/column 6/.style={column sep = -1cm}, /tikz/column 7/.style={column sep = 1cm}]
% & & & & ((g \otimes h) \otimes k) \otimes X\arrow[dllll]\arrow[dd]\arrow[drrr] & & &\\
% (g \otimes (h \otimes k)) \otimes X\arrow[dr]\arrow[dddddd] & & & & & & & (gh \otimes k) \otimes X\arrow[ddl]\arrow[dddddd]\\
% & g \otimes ((h \otimes k) \otimes X)\arrow[dr]\arrow[dddd] & & & (g \otimes h) \otimes (k \otimes X)\arrow[dll]\arrow[d]\arrow[drr] & & &\\
% & & g \otimes (h \otimes (k \otimes X))\arrow[dr] & & (g \otimes h) \otimes X\arrow[dl]\arrow[dr] & & gh \otimes (k \otimes X)\arrow[dl] &\\
% & & & g \otimes (h \otimes X)\arrow[d] & & gh \otimes X\arrow[d] & &\\
% & & & g \otimes X\arrow[rr] & & X & &\\
% & g \otimes (hk \otimes X)\arrow[urr] & & & & & &\\
% (g \otimes hk) \otimes X\arrow[ur]\arrow[rrrrrrr] & & & & & & & ghk \otimes X\arrow[uull]\\
%\end{tikzcd}
%\end{center}
\newpage
\noindent Note that the leftmost pentagon is the pentagon diagram for the associator, while all other pentagons are versions of the pentagon given at the beginning of the proof. The upper middle quadrilateral comes from the functoriality of $\otimes$, with the remaining quadrilaterals come from the naturality of $\alpha$. If we look at the boundary, we obtain the commutative diagram
\begin{center}
\begin{tikzcd}[/tikz/column 1/.style={column sep = 1.5cm}]
(g \otimes h) \otimes k \arrow[dd, "\alpha_{g,h,k}"'] \arrow[r, "\mu_{g,h} \otimes \id_k"] & gh \otimes k \arrow[dr, "\mu_{gh,k}"] \\
& & ghk\\
g \otimes (h \otimes k) \arrow[r, "\id_g \otimes \mu_{h,k}"'] & g \otimes hk \arrow[ur, "\mu_{g,hk}"']
\end{tikzcd}
\end{center}
%\begin{center}
%\begin{tikzcd}
% & (g \otimes h) \otimes k\arrow[ld, "\alpha_{g,h,k}"']\arrow[rd, "\mu_{g,h} \otimes \id_k"] &\\
%g \otimes (h \otimes k)\arrow[d, "\id_g \otimes \mu_{h,k}"'] & & gh \otimes k\arrow[d, "\mu_{gh,k}"]\\
%g \otimes hk\arrow[rr, "\mu_{g,hk}"'] & & ghk
%\end{tikzcd}.
%\end{center}
\noindent Thus the associator for $\mathcal{D}$ is given by
\begin{align*}
\begin{split}
\alpha_{g,h,k} = (\id_g \otimes \mu_{h,k}^{-1}) \circ \mu_{g,hk}^{-1} \circ \mu_{gh,k} \circ (\mu_{g,h} \otimes \id_k),
\end{split}
\end{align*}
\noindent which is cohomologous to the trivial $3$-cocycle. This completes the proof.
\end{proof}\\

\noindent\textcolor{red}{Add citation for Tom's notes.}\\

\noindent\textcolor{red}{We should focus on the case when $G$ is finite Abelian, and possibly when $m = k\abs{G}$ for some $k \in \mathbb{N}$.}\\

\newpage

\noindent Let $\mathcal{C}$ be a near-group category with multiplicity $m$, whose invertible objects form the finite group $G$ of order $n$. Assume this category admits a realization as a subcategory of $\mathcal{E}(A)$ for some algebra $A$. Then we must ask for endomorphisms $\rho$ and $\alpha_g$, for all $g \in G$, satisfying
\begin{align*}
\begin{split}
[\alpha_g] \otimes [\alpha_h] = [\alpha_{gh}],\\
[\alpha_g] \otimes [\rho] = [\rho] \otimes [\alpha_g] = [\rho],\\
[\rho] \otimes [\rho] = \bigoplus_{g \in G} [\alpha_g] \oplus [\rho]^{\oplus m}.
\end{split}
\end{align*}
\noindent %Let $\mathcal{L} \coloneqq \mathcal{L}_{m+n}$ and consider $s_i, s_i', t_g, t_g' \in \mathcal{L}$, for all $i \in \{1, \dots, m\}$ and $g \in G$.
Suppose we choose our $\alpha_g$ such that $\alpha_g \circ \rho = \rho$, for all $g \in G$. Note that we will not necessarily have ``$\rho \circ \alpha_g = \rho$''. By definition, we may only expect $\rho(\alpha_g(x)) = U(g)^{-1}\rho(x)U(g)$ for some invertible $U(g) \in A$. By \hyperref[CommutingInvertibles]{Proposition \ref*{CommutingInvertibles}}, we will however always have $\alpha_g \circ \alpha_h = \alpha_{gh}$ for all $g, h \in G$. The third equation above tells us, by definition of the direct sum, that we must have $s_g, s_g', t_i, t_i' \in A$ such that %for $i \in \{1, \dots, m\}$ and $g \in G$ such that
\begin{align*}
\begin{split}
\rho(\rho(x)) = \sum_{g \in G}{s_g\alpha_g(x)s_g'} + \sum_{i=1}^m{t_i\rho(x)t_i'}
\end{split}
\end{align*}
\noindent for all $x \in A$, where
\begin{align*}
\begin{split}
\sum_{g' \in G}{s_{g'}s_{g'}'} + \sum_{i'=1}^m{t_{i'}t_{i'}'} = 1,\qquad s_g's_h = \delta_{g,h},\qquad t_i't_j = \delta_{i,j}\qquad\text{and}\qquad s_g't_i = 0 = t_i's_g,
\end{split}
\end{align*}
\noindent for all $g, h \in G$ and $i, j \in \{1, \dots, m\}$. However, there are in general no canonical choices for these elements. Our choice of elements $s_e, s_e' t_i, t_i'$ will be unimportant, but we will make a special choice for each $s_g$. In particular, suppose $s_g \coloneqq \alpha_g(s_e)$. Recalling the definition of the monoidal product of morphisms in our category and our assumption that $\alpha_g \otimes \rho = \rho$, this means that
\begin{align*}
\begin{split}
s_g \coloneqq \id_{\alpha_g} \otimes s_e \in \homset_{\mathcal{E}(A)}(g, \rho \otimes \rho).
\end{split}
\end{align*}
\noindent Because $\homset_{\mathcal{E}(A)}(g, \rho \otimes \rho)$ is $1$-dimensional, this is going to be a linear scaling of some injection satisfying the direct sum properties and hence it too will satisfy them. We then have the left inverse
\begin{align*}
\begin{split}
s_g' \coloneqq \id_{\alpha_g} \otimes s_e'
\end{split}
\end{align*}
\noindent for $s_g$. We will now find, for each $g \in G$, an invertible $U(g) \in \homset_{\mathcal{E}(A)}(\rho, \rho \otimes \alpha_g)$ satisfying both $U(g)s_e = s_e$ and $\rho(\alpha_g(x)) = U(g)^{-1}\rho(x)U(g)$, for all $x \in A$. Because $\alpha_g \otimes \rho = \rho$ and the space $\homset_{\mathcal{E}(A)}(\id, \rho \otimes \rho)$ containing $s_e$ is $1$-dimensional, there certainly exists an invertible $U(g)$ satisfying
\begin{align*}
\begin{split}
s_e = (U(g) \otimes \id_\rho) \circ s_e.
\end{split}
\end{align*}
\noindent We also have that, for all $g, h \in G$,
\begin{align*}
\begin{split}
(((U(g) \otimes \id_{\alpha_h}) \circ U(h)) \otimes \id_\rho) \circ s_e = s_e = (U(gh) \otimes \id_\rho) \circ s_e
\end{split}
\end{align*}
\begin{align*}
\begin{split}
\implies (U(g) \otimes \id_{\alpha_h}) \circ U(h) = U(gh)
\end{split}
\end{align*}
\noindent \textcolor{red}{via abstract nonsense}, whence $g \mapsto U(g) \otimes \id_{\rho}$ gives a representation of $G$ on $\homset_{\mathcal{E}(A)}(\rho \otimes \rho, \rho \otimes \rho)$.\\

\noindent\textcolor{red}{Show that $\rho(\alpha_g(x)) = U(g)^{-1}\rho(x)U(g)$.}\\

\newpage

\noindent We would like to show that $\rho$ and $\alpha_g$ restrict to endomorphisms of the Leavitt algebra $\mathcal{L} \coloneqq \mathcal{L}_{m+n}$ with generators $s_g, s_g', t_i, t_i'$, for $g \in G$ and $i \in \{1, \dots, m\}$.\\

\newpage

\noindent Determine how $\rho$ and $\alpha_g$ act on the generators!\\

\newpage

\noindent Suppose we instead have \\

%\noindent Indeed, suppose we fix a non-zero morphism $s_e \in \homset_{\mathcal{E}(\mathcal{L})}(\id, \rho \otimes \rho)$. We claim that this element admits a left inverse. As an intertwiner, $s_e$ is an element satisfying $s_ex = \rho^2(x)s_e$ for all $x \in \mathcal{L}$. Thus $\mathcal{L}s_e$ is a non-zero two-sided ideal. Because $\mathcal{L} \neq \mathcal{L}_1$ is simple, we must have that $\mathcal{L}s_e = \mathcal{L}$. This means that $1 \in \mathcal{L}s_e$, and hence $s_e$ must admit a left inverse. Denote this by $s_e'$. Fix now a non-zero isomorphism $f_g \in \homset_{\mathcal{E}(\mathcal{L})}(\alpha_g \otimes \rho, \rho)$ and let
%\begin{align*}
%\begin{split}
%s_g \coloneqq (f_g \otimes \id_\rho) \circ (\id_{\alpha_g} \otimes s_e) \in \homset_{\mathcal{E}(\mathcal{L})}(\alpha_g, \rho \otimes \rho),
%\end{split}
%\end{align*}
%\noindent for each $g \in G$. Because $s_e$ admits the left inverse $s_e'$, we have also the left inverse
%\begin{align*}
%\begin{split}
%s_g' \coloneqq (\id_{\alpha_g} \otimes s_e') \circ (f_g^{-1} \otimes \id_\rho)
%\end{split}
%\end{align*}
%\noindent for $s_g$.\\

\noindent Because the group-like elements of $\mathcal{C}$ have Frobenius--Perron dimension $1$, the dimension of $\rho$ must satisfy $d^2 = md + n$. In other words, $\rho$ must have dimension
\begin{align*}
\begin{split}
d_\pm \coloneqq \frac{m + \csqrt{m^2 + 4n}}{2}.
\end{split}
\end{align*}
\noindent\\

\newpage

\noindent 

\noindent\textcolor{red}{For each set of fusion rules, find solution set of endomorphisms $\rho$ and $\alpha_g$. Then for each possible solution, show that $\overline{\mathcal{E}(\mathcal{A})}_\oplus$ is a fusion category.}
\newpage

\renewcommand\thesection{R}
\ruledsectionstar{References}{References}
\begingroup
\setlength{\emergencystretch}{.5em}
\printbibliography[heading=none]
\endgroup
\newpage

\end{document}