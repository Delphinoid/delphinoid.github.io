%%%%%%%%%%%%
% DOCUMENT %
%%%%%%%%%%%%

\documentclass[12pt, reqno]{amsart}
\usepackage[a4paper, total={18cm, 22cm}, centering]{geometry}
% Order matters here! The last language is the one that ends up being selected for biber.
\usepackage[russian, japanese, english]{babel}


%%%%%%%%%%%%
% PACKAGES %
%%%%%%%%%%%%

% Miscellaneous packages.
\usepackage{latexsym}
\usepackage{amsmath, amsfonts, amsthm, amssymb, mathrsfs}
\usepackage{graphicx}
\usepackage[usenames, dvipsnames]{color}
%\usepackage[hidelinks]{hyperref}
\usepackage{hyperref}
% Customizable enumerate list labels.
\usepackage{enumerate}
\usepackage[shortlabels]{enumitem}
% Conditional.
\usepackage{ifthen}
\usepackage{etoolbox}

% Tilde notation for vectors.
\usepackage{accents}
\newcommand{\ut}[1]{\underaccent{\tilde}{#1}}
% Arrow notation for vectors.
\usepackage{harpoon}
% Dirac bra-ket notation for quantum states.
\usepackage{braket}
% Other mathematics syntax (like \coloneqq from mathtools and \oiint from esint).
% We also use \DeclarePairedDelimiter from mathtools.
\usepackage{mathtools, esint}
% General blackboard bold symbols.
\usepackage{bbm}

% Scientific notation.
\usepackage{siunitx}
% SI unit formatting.
\sisetup{exponent-product=\ensuremath{\cdot}, inter-unit-product=\ensuremath{\cdot}, separate-uncertainty=true, multi-part-units=single}

% Table colours.
\usepackage[table, xcdraw]{xcolor}
\usepackage{colortbl}
\definecolor{ColourBlack}{HTML}{000000}
\definecolor{ColourWhite}{HTML}{FFFFFF}
\definecolor{ColourTableGrey}{HTML}{EDEDED}
\definecolor{Grey}{gray}{0.92}

% Plotting.
\usepackage{tikz-cd}
\usepackage{spath3}
\usetikzlibrary{arrows.meta, babel, decorations.markings, decorations.pathreplacing, knots}

% Code.
\usepackage{listings}
\definecolor{MyDarkGreen}{rgb}{0.0,0.4,0.0}
\lstloadlanguages{Matlab}
\lstset{language = Matlab,
	frame = single,
	basicstyle = \fontsize{8}{8}\ttfamily,
	keywordstyle = [1]\color{Blue}\bfseries,
	keywordstyle = [2]\color{Purple},
	keywordstyle = [3]\color{Blue}\underbar,
	identifierstyle = ,
	commentstyle = \usefont{T1}{pcr}{}{sl}\color{MyDarkGreen}\small,
	stringstyle = \color{Purple},
	showstringspaces = false,
	tabsize = 4,
	morekeywords = {},
	morekeywords = [2]{},
	morekeywords = [3]{},
	morecomment = [l][\color{Blue}]{...},
	numbers = left,
	firstnumber = 1,
	numberstyle = \tiny\color{Blue},
	stepnumber = 1
}

% Chemistry notation.
\usepackage[version = 4]{mhchem}

% Custom headers and footers.
\usepackage{fancyhdr}
% Makes all pages in the document conform to the custom headers and footers by default.
\pagestyle{fancy}
\fancyfoot[C]{\thepage}
% Remove underlines from headers and footers.
\renewcommand{\headrulewidth}{0pt}
\renewcommand{\footrulewidth}{0pt}
\setlength{\headheight}{14pt}
\setlength{\footskip}{14pt}

% Bibliography.
\usepackage{doi}
% Use the custom style "math-alphabetic" for nicer references.
\usepackage[
	backend=biber,
	style=math-alphabetic,
	firstinits=true,
	dashed=false,
	url=false,
	doi=true,
	minalphanames=3,
	maxalphanames=4,
	maxnames=4,
	sorting=none
]{biblatex}
% Use normal font size for references.
\renewcommand*{\bibfont}{\normalsize}
% Use title case rather than sentence case for references.
\DeclareFieldFormat{titlecase}{#1}
% Specify the bibliography data file to use.
\addbibresource{References.bib}
% Use "Ph.D. thesis" instead of "PhD thesis".
\DefineBibliographyStrings{english}{phdthesis = {Ph\adddot D\adddotspace thesis}}
% Put last names first.
\DeclareNameAlias{default}{family-given}
% Sort by order present in .bib file.
\nocite{*}

% Date formatting for declaration.
\usepackage[ddmmyyyy]{datetime}

% Prevent hyphenation.
\usepackage[none]{hyphenat}

% Center the title.
\usepackage{titling}
\renewcommand\maketitlehooka{\null\mbox{}\vfill}
\renewcommand\maketitlehookd{\vfill\null}


%%%%%%%%%%%%
% COMMANDS %
%%%%%%%%%%%%

% Negative horizontal phantom.
\newcommand{\nhphantom}[1]{\sbox0{#1}\hspace{-\the\wd0}}


%%%%%%%%%%%%
% NOTATION %
%%%%%%%%%%%%

% Differential formatting.
\newcommand*{\ndiff}[1]{\mathrm{d}#1}
\newcommand*{\sdiff}[1]{\mathop{}\!\ndiff{#1}}
\newcommand{\rdiff}[3][]{
	\ifthenelse{\equal{#1}{}}
		{\frac{\mathrm{d}#2}{\mathrm{d}#3}}
		{\frac{\mathrm{d}^{#1}#2}{\forcsvlist\ndiff{#3}}}
}
\newcommand*{\npiff}[1]{\mathrm{\partial}#1}
\newcommand*{\spiff}[1]{\mathop{}\!\npiff{#1}}
\newcommand{\rpiff}[3][]{
	\ifthenelse{\equal{#1}{}}
		{\frac{\mathrm{\partial}#2}{\mathrm{\partial}#3}}
		{\frac{\mathrm{\partial}^{#1}#2}{\forcsvlist\npiff{#3}}}
}
% Inexact differential for physics.
\newcommand*{\dbar}[1]{\mathop{}\!\mathrm{\dj}#1}

% Metrics, inner products and norms.
\DeclarePairedDelimiter{\abs}{\lvert}{\rvert}
\DeclarePairedDelimiter{\inprod}{\langle}{\rangle}
\DeclarePairedDelimiter{\norm}{\lVert}{\rVert}
% This is used if we want an empty norm. 
\newcommand{\blank}{{}\cdot{}}
% Floor and ceiling functions.
\DeclarePairedDelimiter{\floor}{\lfloor}{\rfloor}
\DeclarePairedDelimiter{\ceil}{\lceil}{\rceil}

% Function notation.
\newcommand{\id}{\textup{id}}
\newcommand{\coker}{\textup{Coker}}
\newcommand{\im}{\textup{Im}}
\newcommand{\ev}{\textup{ev}}
\newcommand{\coev}{\textup{coev}}

% Category theory notation.
%\newcommand{\obset}{\textup{Ob}_{#1}\!\left(#2\right)}
%\newcommand{\morset}[2][]{\textup{Mor}_{#1}\!\left(#2\right)}
%\newcommand{\homset}[2][]{\textup{Hom}_{#1}\!\left(#2\right)}
%\newcommand{\End}[2][]{\textup{End}_{#1}\!\left(#2\right)}
%\newcommand{\opp}[1]{{#1}^{\textup{op}}}
%\newcommand{\textcat}[1]{\textup{\textsf{#1}}}
\newcommand{\obset}{\textup{Ob}}
\newcommand{\morset}{\textup{Mor}}
\newcommand{\homset}{\textup{Hom}}
\newcommand{\End}{\textup{End}}
\newcommand{\opp}{\textup{op}}
\newcommand{\textcat}[1]{\textup{\textsf{#1}}}
\newcommand{\textobj}[1]{\textup{\texttt{#1}}}

% Special notation.
%\newcommand{\chr}{\textup{char}}
%\newcommand{\Tr}{\textup{Tr}}
%\newcommand{\trv}{\textup{tr}}
%\newcommand{\Dim}{\textup{Dim}}
%\newcommand{\FPdim}{\textup{FPdim}}
\newcommand{\chr}{\textup{char}}
\newcommand{\Tr}{\textup{Tr}}
\newcommand{\trv}{\textup{tr}}
\newcommand{\Dim}{\textup{Dim}}
\newcommand{\FPdim}{\textup{FPdim}}

% Hiragana "yo" for the Yoneda embeddings.
\newcommand{\yo}{\text{\usefont{U}{min}{m}{n}\symbol{'210}}}
\DeclareFontFamily{U}{min}{}
\DeclareFontShape{U}{min}{m}{n}{<-> udmj30}{}

% Representation theory notation.
\newcommand{\Sym}{\textup{Sym}}
\newcommand{\Alt}{\textup{Alt}}

% Natural isomorphism arrows.
\usetikzlibrary{nfold}

% Long squiggly arrow.
\usetikzlibrary{decorations.pathmorphing}
\newcommand\xlsquigarrow[1]{\mathrel{
	\begin{tikzpicture}[baseline={(current bounding box.south)}]
		\node[inner sep=.44ex,align=center] (tmp) {$\scriptstyle #1$};
		\path[draw,<-,decorate,decoration={zigzag,amplitude=1pt,segment length=1.75mm,pre length=3.5pt}] (tmp.south west) -- (tmp.south east);
	\end{tikzpicture}
}}
\newcommand\xrsquigarrow[1]{\mathrel{
	\begin{tikzpicture}[baseline={(current bounding box.south)}]
		\node[inner sep=.44ex,align=center] (tmp) {$\scriptstyle #1$};
		\path[draw,<-,decorate,decoration={zigzag,amplitude=-1pt,segment length=1.75mm,pre length=3.5pt}] (tmp.south east) -- (tmp.south west);
	\end{tikzpicture}
}}
\newcommand\xlrsquigarrow[1]{\mathrel{
	\begin{tikzpicture}[baseline={(current bounding box.south)}]
		\node[inner sep=.44ex,align=center] (tmp) {$\scriptstyle #1$};
		\path[draw,<->,decorate,decoration={zigzag,amplitude=-1pt,segment length=1.75mm,pre length=3.5pt, post length=2.75pt}] (tmp.south east) -- (tmp.south west);
	\end{tikzpicture}
}}

% Closed square root.
\def\DHLhksqrt[#1]#2#3{
	\setbox0=\hbox{\nhphantom{$#2\sqrt[#1]{#3\,}$}\phantom{$#2\sqrt{#3\,}$}$#2\sqrt[#1]{#3\,}$}\dimen0=\ht0
	\advance\dimen0-0.4\ht0
	\setbox2=\hbox{\vrule height\ht0 depth -\dimen0}
	{\box0\lower0.5pt\box2}
}
\newcommand\csqrt[2][]{\mathchoice
	{\DHLhksqrt[#1]{\displaystyle}{#2}}
	{\DHLhksqrt[#1]{\textstyle}{#2}}
	{\DHLhksqrt[#1]{\scriptstyle}{#2}}
	{\DHLhksqrt[#1]{\scriptscriptstyle}{#2}}
}

% More friendly Fraktur S and P.
\newcommand{\mathfrakS}{\mathpalette\bigmathfrakS\relax}
\newcommand{\bigmathfrakS}[2]{\scalebox{1.6}{$#1\mathfrak{s}$}}
\newcommand{\mathfrakP}{\mathpalette\bigmathfrakP\relax}
\newcommand{\bigmathfrakP}[2]{\scalebox{1.3}{$#1\mathfrak{p}$}}

% Inline limits, etc.
\newcommand\inlinestack[1]{\mathop{\vcenter{\hbox{${#1}$}}}}

% Warning symbol.
\newcommand{\warning}{\raisebox{2.65pt}{\fontencoding{U}\fontfamily{futs}\selectfont\char 49\relax}}

% Use symbols instead of numbers for footnotes.
% 0. 
% 1. *
% 2. †
% 3. ‡
% 4. §
% 5. ¶
% 6. ‖
% 7. **
% 8. ††
% 9. ‡‡
\renewcommand{\thefootnote}{\fnsymbol{footnote}}


%%%%%%%%%%%%%%
% FORMATTING %
%%%%%%%%%%%%%%

% Bold subsection formatting.
%\makeatletter
%\def\@seccntformat#1{
%	\protect\@secnumfont
%	\expandafter\protect\csname format#1\endcsname
%	\csname the#1\endcsname
%	\protect\@secnumpunct
%}
%\newcommand{\formatsection}{\bfseries\boldmath}
%\newcommand{\formatsubsection}{\bfseries\boldmath}
%\newcommand{\formatsubsubsection}{\bfseries\boldmath}

% Section formatting.
\numberwithin{figure}{section}
\numberwithin{table}{section}
\makeatletter
\def\section{
	\@startsection{section}{2}
	\z@{0pt}{.5\linespacing}{\Large\scshape}%\bfseries\boldmath}
}
\def\subsection{
	\@startsection{subsection}{2}
	\z@{0pt}{-.5em}{\normalfont}%\bfseries\boldmath}
}
\makeatother

% Number equations according to the sections they appear in.
%\numberwithin{equation}{section}
%\numberwithin{equation}{subsection}
%\numberwithin{equation}{subsubsection}

% Ruled sections.
\def\ruledsection#1#2{\section{#1}\label{#2}\noindent\\[-1.75\linespacing]\rule{\linewidth}{1pt}\\[-0.75\linespacing]}
\def\ruledsectionstar#1#2{\section*{#1}\label{#2}\noindent\\[-1.75\linespacing]\rule{\linewidth}{1pt}\\[-0.75\linespacing]}

% Remove the ugly random spacing that LaTeX likes to add.
\raggedbottom


%%%%%%%%%%%%%%%%
% ENVIRONMENTS %
%%%%%%%%%%%%%%%%

% Sections that support equation numbering.
\newcommand*{\problem}[2][]{
	\ifthenelse{\equal{#1}{}}
		{\section*{#2}\refstepcounter{section}}
		{\section*{\texorpdfstring{#2}{#1}}\refstepcounter{section}}
}
\newcommand*{\subproblem}[2][]{
	\ifthenelse{\equal{#1}{}}
		{\subsection*{#2}\refstepcounter{subsection}}
		{\subsection*{\texorpdfstring{#2}{#1}}\refstepcounter{subsection}}
}
\newcommand*{\subsubproblem}[2][]{
	\ifthenelse{\equal{#1}{}}
		{\subsubsection*{#2}\refstepcounter{subsubsection}}
		{\subsubsection*{\texorpdfstring{#2}{#1}}\refstepcounter{subsubsection}}
}

% Theorem environments.
\newtheoremstyle{_plain}{-\topsep}{-\topsep}{\itshape}{}{\bfseries}{.}{0.5em}{}
\theoremstyle{_plain}
\newtheorem{theorem}{Theorem}[section]
\newtheorem{theoremdefinition}[theorem]{Theorem-Definition}
\newtheorem{lemma}[theorem]{Lemma}
\newtheorem{lemmadefinition}[theorem]{Lemma-Definition}
\newtheorem{proposition}[theorem]{Proposition}
\newtheorem{propositiondefinition}[theorem]{Proposition-Definition}
\newtheorem{corollary}[theorem]{Corollary}
\newtheorem{corollarydefinition}[theorem]{Corollary-Definition}
\newtheorem{conjecture}[theorem]{Conjecture}
\newtheorem{definition}[theorem]{Definition}

\newtheoremstyle{_definition}{-\topsep}{-\topsep}{}{}{\bfseries}{.}{0.5em}{}
\theoremstyle{_definition}
\newtheorem{example}[theorem]{Example}

%\theoremstyle{remark}
\newtheorem{remark}[theorem]{Remark}
\newtheorem{intuition}[theorem]{Intuition}
\newtheorem{daggerremark}[theorem]{$\dagger$ Remark}
\newtheorem{question}[theorem]{Question}
\newtheorem{notation}[theorem]{Notation}
\newtheorem{observation}[theorem]{Observation}

% Proof environment formatting.
\newcommand{\tombstone}{\ensuremath{\blacksquare}}
\newcommand{\sketchstone}{\ensuremath{\square}}
\renewenvironment{proof}{{\textbf{Proof.}}}{\null\hfill\tombstone}
\newenvironment{sketch}{{\textit{Sketch.}}}{\null\hfill\sketchstone}

% Modified matrix environments for augmented matrices.
\makeatletter
\renewcommand*\env@matrix[1][*\c@MaxMatrixCols c]{
	\hskip -\arraycolsep
	\let\@ifnextchar\new@ifnextchar
	\array{#1}
}
\makeatother

% Modified cases environment for column alignment.
\makeatletter
\renewenvironment{cases}[1][l]{\matrix@check\cases\env@cases{#1}}{\endarray\right.}
\def\env@cases#1{
	\let\@ifnextchar\new@ifnextchar
	\left\lbrace\def\arraystretch{1.2}
	\array{@{}#1@{\quad}l@{}}}
\makeatother