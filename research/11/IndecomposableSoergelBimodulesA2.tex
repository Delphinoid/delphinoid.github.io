\input{preamble.tex}

\begin{document}

\thispagestyle{fancy}

\begin{center}
\LARGE\scshape Indecomposable Soergel Bimodules\linebreak of Types $A_1$ and $A_2$\noindent\\[-\linespacing]
\rule{0.75\linewidth}{1pt}
\end{center}
\noindent\\[-0.75\linespacing]

%\ruledsection{Prologue}{1}
%\noindent\\ 

\noindent The goal of these notes is to explicitly step through the classification of the indecomposable Soergel bimodules of type $A_2$, while also covering the simpler $A_1$ case as a stepping stone. Throughout these notes, $\mathbbm{k}$ will always be taken to be an algebraically closed field of characteristic zero.\\

\noindent\begin{definition}\label{GeometricRepresentation}\textup{(Geometric Representation).} Let $(W, S)$ be a Coxeter system and $V$ the $\mathbbm{k}$-vector space with formal basis $\{\alpha_s : s \in S\}$. Define a symmetric, bilinear form on $V$ by linearly extending
\begin{align*}
\begin{split}
(\alpha_s, \alpha_t) = \begin{cases}-\cos\!\left(\frac{\pi}{m_{st}}\right)\!,&m_{st} \neq \infty;\\-1,&m_{st} = \infty.\end{cases}
\end{split}
\end{align*}
\noindent From this, we define an action of $s \in S$ on the basis elements $\alpha_t \in V$ by the linear automorphism
\begin{align*}
\begin{split}
s(\alpha_t) \coloneqq \alpha_t - 2(\alpha_s, \alpha_t)\alpha_s,
\end{split}
\end{align*}
\noindent which reflects $\alpha_t$ across $\alpha_s$. The {\em geometric representation} of $(W, S)$ is the representation induced by linearly extending this reflection action to an action of $W$ on all of $V$.\\
\end{definition}

\noindent Recall that by definition, $R \coloneqq \bigoplus_{i=0}^\infty\textup{Sym}^i(V)$, where $\textup{Sym}^i(V)$ is the quotient of the $i$th tensor power $V^{\otimes i}$ by the action of the symmetric group $S_i$, viewed as a $\mathbbm{k}$-module that is $\mathbb{Z}$-graded in degree $2$. Our reflection action also extends to an action on $R$ by taking $s(fg) \coloneqq s(f)s(g)$, for $f, g \in R$.\linebreak It is not too difficult to show that $\textup{Sym}^i(V)$ is isomorphic to the additive subgroup of the polynomial ring $\mathbbm{k}[\alpha_s : s \in S]$ consisting only of homogeneous polynomials of degree $i$, meaning we may identify $R = \mathbbm{k}[\alpha_s : s \in S]$. With this in mind, we have a crucial lemma that we will invoke regularly.\\

\noindent\begin{lemma}\label{HomogeneousGenerator} Suppose that $M$ is a graded $(R, R)$-bimodule that is generated by a homogeneous element $m \in M$, in the sense that $M = RmR$. Then $M$ is indecomposable.\\
\end{lemma}

\noindent\begin{proof} Let $d$ denote the degree of $m$. Because $M$ is generated by $m$, and because $R^0 = \mathbbm{k}$, we have that $M^d = R^0mR^0 = \mathbbm{k}m$. In other words, $M^d$ is a one-dimensional vector space. Suppose that $M \cong L \oplus N$. In this case, $M^d$ is isomorphic as a $\mathbbm{k}$-vector space to $L^d \oplus N^d$. Assume without loss of generality that $m \in L^d$. This forces $N^d = 0$, as $M^d$ is one-dimensional, whence we have that $M = RmR \subseteq L$, forcing $N = 0$. Thus $M$ is indecomposable. This completes the proof.
\end{proof}\\

\noindent From this lemma, it follows that $R$ itself is indecomposable as an $(R, R)$-bimodule, as it is generated by $1 \in \mathbbm{k}$. Moreover, suppose we define
\begin{align*}
\begin{split}
B_s \coloneqq R \otimes_{R^s} R(1)
\end{split}
\end{align*}
\noindent for any $s \in S$, where $R^s \coloneqq \{f \in R : s(f) = f\}$. This is also clearly indecomposable by our lemma, as it is generated as an $(R, R)$-bimodule by $1 \otimes_{R^s} 1$ (where we note that, since $1 \in R^0 = R(1)^{-1}$, we have $1 \otimes_{R^s} 1 \in B_s^{0-1} = B_s^{-1}$, meaning it is homogeneous of degree $-1$). In particular, it is easy to see that grading shifts of $R$ and $B_s$ are all indecomposable too.\newpage

\noindent\begin{definition}\textup{(Soergel Bimodule).} Let $(W, S)$ be a Coxeter system and $\underline{w} \coloneqq (s_1, \dots, s_k)$ an expression. The {\em Bott--Samelson bimodule corresponding to $\underline{w}$} is the graded $(R, R)$-bimodule
\begin{align*}
\begin{split}
BS(\underline{w}) &\coloneqq B_{s_1} \otimes_R \cdots \otimes_R B_{s_k}\\
&\hphantom{\coloneqq}\nhphantom{=}= (R \otimes_{R^{s_1}} R(1)) \otimes_R \cdots \otimes_R (R \otimes_{R^{s_k}} R(1))\\
&\hphantom{\coloneqq}\nhphantom{$\cong$}\cong R \otimes_{R^{s_1}} R \otimes_{R^{s_2}} \cdots \otimes_{R^{s_k}} R(k).
\end{split}
\end{align*}
\noindent A {\em Soergel bimodule} is any graded $(R, R)$-bimodule that is isomorphic to a finite direct sum of grading shifs of direct summands of Bott--Samelson bimodules.\\
\end{definition}

%\noindent Given $s \in S$, let $\partial_s : f \mapsto \frac{f - s(f)}{\alpha_s}$ for $f \in R$ be the corresponding {\em Demazure operator}. To see that this is well-defined, we have the following useful lemma, which shows us that any polynomial in $R$ can in fact be split into the sum of an $s$-invariant component and an $s$-antiinvariant component. \\

%\noindent\begin{lemma}\label{RSplitting} For all $s \in S$ and $f \in R$, we have $f = \frac{1}{2}(\partial_s(f\alpha_s) + \partial_s(f)\alpha_s)$, where $\partial_s(f\alpha_s), \partial_s(f) \in R^s$. In particular, we have an $(R^s, R^s)$-bimodule isomorphism $R \cong R^s \oplus R^s\alpha_s \cong R^s \oplus R^s(-2)$.\\
%\end{lemma}

%\noindent\begin{proof} Let $s \in S$ and fix $t, u \in S$. The Demazure operator is certainly well-defined on $\alpha_t$, since $\alpha_t - s(\alpha_t) = 2(\alpha_s, \alpha_t)\alpha_s \in R^s\alpha_s$. By linearity we have that $\partial_s$ is well-defined on sums, and since $\alpha_t\alpha_u - s(\alpha_t\alpha_u) = (2(\alpha_s, \alpha_u)\alpha_t + 2(\alpha_s, \alpha_t)\alpha_u + 4(\alpha_s, \alpha_t)(\alpha_s, \alpha_u)\alpha_s)\alpha_s$ it is also well-defined on products. Thus it is well-defined on all of $R$. Since $\partial_s(f\alpha_s) = f + s(f) = 2f - 2($, it is easy to see that
%Let $s \in S$ and $f \coloneqq \alpha_t$ for any $t \in S$. Since $f - s(f) = 2(\alpha_s, \alpha_t)\alpha_s \in R^s\alpha_s$, the Demazure operator is clearly well-defined in, and $\partial_s(f\alpha_s) = f + s(a_t)
%Observe that $\rho_s(f) \coloneqq \frac{1}{2}(f + s(f)) = \alpha_t - (\alpha_s, \alpha_t)\alpha_s \in R^s$, while $\partial_s(f)\alpha_s \coloneqq \frac{1}{2}(f - s(f)) = (\alpha_s, \alpha_t)\alpha_s \in R^s \alpha_s$. Naturally sums will preserve this splitting, so all that remains is to show that products preserve it too. Suppose $f \coloneqq f_1 + f_2\alpha_s$ and $g \coloneqq g_1 + g_2\alpha_s$, for $f_1, f_2, g_1, g_2 \in R^s$. %, for $g_1, h_1, g_2, h_2 \in R^s$.
%Then ${fg = f_1g_1 + f_1g_2\alpha_s + f_2g_1\alpha_s + f_2g_2\alpha_s^2}$ admits a unique decomposition of the form $P_s(fg) + \partial_s(fg)\alpha_s$, where $\rho_s(fg) = g_1g_2 + h_1h_2\alpha_s^2$ and $\partial_s(fg) = g_1h_2\alpha_s + g_2h_1\alpha_s$.\linebreak This completes the proof.
%\end{proof}\\

\noindent Another useful lemma is that any polynomial in $R$ can be split into the sum of an $s$-invariant component and an $s$-antiinvariant component, for any $s \in S$.\\

\noindent\begin{lemma}\label{RSplitting} For any $s \in S$ and $f \in R$, we have that $f + s(f) \in R^s$ and $f - s(f) \in R^s\alpha_s$. In particular, we have a graded $(R^s, R^s)$-bimodule splitting $R \cong R^s \oplus R^s\alpha_s \cong R^s \oplus R^s(-2)$ for all $s \in S$.\\
\end{lemma}

\noindent\begin{proof} Let $s \in S$ and $f \coloneqq \alpha_t$ for any $t \in S$. Observe that $P_s(f) \coloneqq \frac{1}{2}(f + s(f)) = \alpha_t - (\alpha_s, \alpha_t)\alpha_s$ is in $R^s$ (alternatively, $s(f + s(f)) = s(f) + s^2(f) = s(f) + f$). Similarly, ${\partial_s(f)\alpha_s \coloneqq \frac{1}{2}(f - s(f)) = (\alpha_s, \alpha_t)\alpha_s}$ is clearly in $R^s \alpha_s$ (alternatively, $s(f - s(f)) = s(f) - s^2(f) = s(f) - f$). By linearity of the action of $S$, sums will preserve this splitting, so all that remains is to show that products preserve it too. Suppose $f \coloneqq f_1 + f_2\alpha_s$ and $g \coloneqq g_1 + g_2\alpha_s$, for $f_1, f_2, g_1, g_2 \in R^s$, such that ${fg = f_1g_1 + f_1g_2\alpha_s + f_2g_1\alpha_s + f_2g_2\alpha_s^2}$. We see that $P_s(fg) = g_1g_2 + h_1h_2\alpha_s^2$ and $\partial_s(fg)\alpha_s = g_1h_2\alpha_s + g_2h_1\alpha_s$, whence $fg$ admits a unique decomposition of the form $fg = P_s(fg) + \partial_s(fg)\alpha_s$. Clearly this induces an isomorphism of graded $(R^s, R^s)$-bimodules, as for any $f \in R$ and $g, h \in R^s$, we have that $hfg \mapsto (hP_s(f)g, h\partial_s(f)\alpha_s g)$.\linebreak This completes the proof.
\end{proof}\\

\noindent Note that this is an isomorphism of $(R^s, R^s)$-bimodules, {\em not} $(R, R)$-bimodules! This lemma allows us to completely classify the indecomposables in the type $A_1$ case.\\

\noindent\begin{proposition}\textup{(Indecomposable Soergel Bimodules of Type $A_1$).} The indecomposable Soergel bimodules of type $A_1$ are, up to grading shift and isomorphism, $R$ and $B_s$.\\
\end{proposition}

\noindent\begin{proof} Let $W = S_2 = \langle s\rangle$ and $S = \{s\}$. By our previous observations, we know that grading shifts of $R$ and $B_s$ are indecomposable, as they are generated by the homogeneous unit tensors. The next step is to check the Bott--Samelson $B_s \otimes_R B_s$. But observe that
\begin{align*}
B_s \otimes_R B_s &= (R \otimes_{R^s} R(1)) \otimes_R (R \otimes_{R^s} R(1))\\
&\cong R \otimes_{R^s} R \otimes_{R^s} R(2)\\
&\cong R \otimes_{R^s} (R^s \oplus R^s(-2)) \otimes_{R^s} R(2) \tag*{(\hyperref[RSplitting]{Lemma \ref*{RSplitting}})}\\
&\cong (R \otimes_{R^s} R(2)) \oplus (R \otimes_{R^s} R)\\
&= B_s(1) \oplus B_s(-1).
\end{align*}
\noindent Thus $B_s \otimes_R B_s$ is decomposable, meaning that we have exhausted all candidates for indecomposables. This completes the proof.
\end{proof}\\

\noindent\begin{remark} Observe that the indecomposables are, up to grading shift, in bijection with $W$, just as we would expect by the categorification theorem.\\
\end{remark}

\noindent One last lemma that we will find useful before we attempt to classify the indecomposable Soergel bimodules of type $A_2$ is the following.\\

\noindent\begin{lemma}\label{RSplitting2} Let $s, t \in S$ with $s \neq t$. Then $R^s$ and $R^t$ generate $R$ as a ring if and only if $m_{st} \neq \infty$.\\
\end{lemma}

\noindent\begin{proof} Recall that $R \coloneqq \textup{Sym}(V)$. As a polynomial ring, $R$ is generated by its linear terms -- those being the vectors in $V$. The linear terms in $R^s$ are precisely the vectors fixed by $s$. This ``intersection of $R^s$ with $V$'' is nothing but
\begin{align*}
\begin{split}
H_s \coloneqq \{v \in V : s(v) = v\} = \{v \in V : (v, \alpha_s) = 0\}.
\end{split}
\end{align*}
Note that the map $\varphi_s : V \to \mathbbm{k}$ given by $v \mapsto (v, \alpha_s)$ is obviously a surjective linear map, as $\varphi_s(\lambda\alpha_s)$ always maps to $\lambda \in \mathbbm{k}$. We therefore have that
\begin{align*}
\begin{split}
\dim(H_s) = \dim(\ker(\varphi_s)) = \dim(V)-\dim(\textup{rank}(\varphi_s)) = \dim(V)-1
\end{split}
\end{align*}
\noindent by the rank-nullity theorem. In other words, $H_s$ is a hyperplane. When $m_{st} \neq \infty$, we have that $H_s$ and $H_t$ are distinct. Because they are hyperplanes, this means that they must span $V$, whence $R^s$ and $R^t$ must generate $R$. This completes the proof.
%Note that, for any $s \in S$, the ring $R^s$ is generated by the unit, $\alpha_s^2$ and elements of the form $\alpha_t - (\alpha_s, \alpha_t)\alpha_s$, for all $t \in S\setminus\{s\}$. Suppose we fix $s, t \in S$ with $s \neq t$. If $m_{st} \neq \infty$, then
%\begin{align*}
%\begin{split}
%\alpha_s = \frac{1}{(\alpha_s, \alpha_t)^2 + 1}(\alpha_s - (\alpha_s, \alpha_t)\alpha_t) + \frac{(\alpha_s, \alpha_t)}{(\alpha_s, \alpha_t)^2 + 1}(\alpha_t - (\alpha_s, \alpha_t)\alpha_s),
%\end{split}
%\end{align*}
%\noindent and similarly for $\alpha_t$. In other words, the subrings $R^s$ and $R^t$ generate the entirety of $R$ as a ring.\linebreak If $m_{st} = \infty$, it is easy to see that we can obtain neither $\alpha_s$ nor $\alpha_t$. This completes the proof.
\end{proof}\\

\noindent\begin{remark} This result becomes trivial when considering that, for any $s \in S$, the ring $R^s$ is generated by the unit, $\alpha_s^2$ and elements of the form $\alpha_t - (\alpha_s, \alpha_t)\alpha_s$, for all $t \in S\setminus\{s\}$. It is easy to see that, given fixed $s, t \in S$ with $s \neq t$, we have
\begin{align*}
\begin{split}
\alpha_s = \frac{1}{(\alpha_s, \alpha_t)^2 + 1}(\alpha_s - (\alpha_s, \alpha_t)\alpha_t) + \frac{(\alpha_s, \alpha_t)}{(\alpha_s, \alpha_t)^2 + 1}(\alpha_t - (\alpha_s, \alpha_t)\alpha_s)
\end{split}
\end{align*}
\noindent if and only if $m_{st} \neq \infty$, and similarly for $\alpha_t$. However, showing that these polynomials generate $R^s$ is non-trivial, following from the Chevalley--Shephard--Todd theorem.\\%Another way to see this is as follows. Recall that $R \coloneqq \textup{Sym}(V)$. As a polynomial ring, $R$ is generated by its linear terms -- those being the vectors in $V$. The linear terms in $R^s$ are precisely the vectors fixed by $s$. This ``part of $R^s$ that lies in V'' is nothing but
%\begin{align*}
%\begin{split}
%H_s \coloneqq \{v \in V : s(v) = v\} = \{v \in V : (v, \alpha_s) = 0\}.
%\end{split}
%\end{align*}
%Note that the map $\varphi_s : V \to \mathbbm{k}$ given by $v \mapsto (v, \alpha_s)$ is obviously a surjective linear map, as $\varphi_s(\lambda\alpha_s)$ always maps to $\lambda \in \mathbbm{k}$. We therefore have that
%\begin{align*}
%\begin{split}
%\dim(H_s) = \dim(\ker(\varphi_s)) = \dim(V)-\dim(\textup{rank}(\varphi_s)) = \dim(V)-1
%\end{split}
%\end{align*}
%\noindent by the rank-nullity theorem. In other words, $H_s$ is a hyperplane. When $m_{st} \neq \infty$, we have that $H_s$ and $H_t$ are distinct. Because they are hyperplanes, this means that they must span $V$, whence $R^s$ and $R^t$ must generate $R$.\\
\end{remark}

\noindent Let $s, t \in S$ with $s \neq t$ and $m_{st} \neq \infty$. Observe that
\begin{align*}
\begin{split}
B_s \otimes_R B_t \cong R \otimes_{R^s} R \otimes_{R^t} R(2)\qquad\text{and}\qquad B_t \otimes_R B_s \cong R \otimes_{R^t} R \otimes_{R^s} R(2).
\end{split}
\end{align*}
\noindent It follows from \hyperref[HomogeneousGenerator]{Lemma \ref*{HomogeneousGenerator}} that these are both indecomposable, as by \hyperref[RSplitting2]{Lemma \ref*{RSplitting2}} they are generated by the degree $-2$ elements $1 \otimes_{R^s} 1 \otimes_{R^t} 1$ and $1 \otimes_{R^t} 1 \otimes_{R^s} 1$, respectively. We shall therefore write $B_{st} \coloneqq B_s \otimes_R B_t$ and $B_{ts} \coloneqq B_t \otimes_R B_s$ from this point onwards.\\

\noindent\begin{remark} As it happens, $B_s \otimes_R B_t$ and $B_t \otimes_R B_s$ remain indecomposable when $m_{st} = \infty$, but proving this is somewhat more involved. In addition to our previous lemma no longer holding (since $H_s = H_t$ end up being the same $1$-dimensional space, generated by $\alpha_s + \alpha_t$), these bimodules are also no longer cyclic, meaning we cannot use \hyperref[HomogeneousGenerator]{Lemma \ref*{HomogeneousGenerator}}!\\
\end{remark}

\noindent The remainder of these notes will be in proving the following result, where $B_{sts} \coloneqq R \otimes_{R^{s,t}} R(3)$.\linebreak This notation will be justified shortly.\\

\noindent\begin{proposition}\textup{(Indecomposable Soergel Bimodules of Type $A_1$).} The indecomposable Soergel bimodules of type $A_2$ are, up to grading shift and isomorphism, $R$, $B_s$, $B_t$, $B_{st}$, $B_{ts}$ and $B_{sts}$.\\
\end{proposition}

\noindent\begin{proof} Let $W = S_3 = \{1, s, t, st, ts, sts\}$ and $S = \{s, t\}$. We have already seen that $R$, $B_s$, $B_t$, $B_{st}$ and $B_{ts}$ are indecomposable Soergel bimodules. Certainly $B_{sts}$ is indecomposable as an $(R, R)$-bimodule by \hyperref[HomogeneousGenerator]{Lemma \ref*{HomogeneousGenerator}}, as it is generated by the degree $-3$ element $1 \otimes_{R^{s,t}} 1$, but it remains to be shown that it is indeed a Soergel bimodule. We will begin by showing that it appears as a direct summand of both $B_s \otimes_R B_t \otimes_R B_s$ and $B_t \otimes_R B_s \otimes_R B_t$. Once we've done this, we will show that these indecomposables are exhaustive.\\[-1.5\baselineskip]
\begin{center}
\rule{0.5\linewidth}{1pt}
\end{center}
\noindent\\[-\baselineskip] In order to show that $B_{sts}$ is a Soergel bimodule, we first claim that
\begin{align*}
\begin{split}
B_s \otimes_R B_t \otimes_R B_s \cong B_{sts} \oplus B_s\qquad\text{and}\qquad B_t \otimes_R B_s \otimes_R B_t \cong B_{sts} \oplus B_t.
\end{split}
\end{align*}
\noindent To this end, let's define an $(R, R)$-bimodule homomorphism $\phi : B_{sts} \to B_s \otimes_R B_t \otimes_R B_s$ by extending
\begin{align*}
\begin{split}
\phi : 1 \otimes_{R^{s,t}} 1 \mapsto 1 \otimes_{R^s} 1 \otimes_{R^t} 1 \otimes_{R^s} 1.
\end{split}
\end{align*}
\noindent This is clearly a well-defined homomorphism of graded $(R, R)$-bimodules, as $f \otimes_{R^{s,t}} 1 = 1 \otimes_{R^{s,t}} f$ if and only if $f \in R^{s,t}$ if and only if $f \in R^s, R^t$. In fact, this reasoning also shows that it is injective. Suppose we define another $(R, R)$-bimodule homomorphism $\psi : B_s \to B_s \otimes_R B_t \otimes_R B_s$ by extending
\begin{align*}
\begin{split}
\psi : 1 \otimes_{R^s} 1 &\mapsto \frac{1}{2}(1 \otimes_{R^s} (\alpha_t \otimes_{R^t} 1 + 1 \otimes_{R^t} \alpha_t) \otimes_{R^s} 1).\\
% &\hphantom{\mapsto}\nhphantom{$=$}= \frac{1}{2}(1 \otimes_{R^s} \alpha_t \otimes_{R^t} 1 \otimes_{R^s} 1 + 1 \otimes_{R^s} 1 \otimes_{R^t} \alpha_t \otimes_{R^s} 1),
\end{split}
\end{align*}
\noindent In order to show that this is well-defined, we need to show that
\begin{align*}
\begin{split}
f \otimes_{R^s} (\alpha_t \otimes_{R^t} 1 + 1 \otimes_{R^t} \alpha_t) \otimes_{R^s} 1 = 1 \otimes_{R^s} (\alpha_t \otimes_{R^t} 1 + 1 \otimes_{R^t} \alpha_t) \otimes_{R^s} f
\end{split}
\end{align*}
\noindent for all $f \in R^s$. By \hyperref[RSplitting]{Lemma \ref*{RSplitting}}, for any $f \in R$ we have
\begin{align*}
\begin{split}
f = P_t(f) + \partial_t(f)\alpha_t,
\end{split}
\end{align*}
\noindent where $P_t(f), \partial_t(f) \in R^t$. If $f \in R^s$, then
\begin{align*}
\begin{split}
f \otimes_{R^s} \alpha_t \otimes_{R^t} 1 \otimes_{R^s} 1 &= 1 \otimes_{R^s} f\alpha_t \otimes_{R^t} 1 \otimes_{R^s} 1\\
&= 1 \otimes_{R^s} (P_t(f) + \partial_t(f)\alpha_t)\alpha_t \otimes_{R^t} 1 \otimes_{R^s} 1\\
&= (1 \otimes_{R^s} P_t(f)\alpha_t \otimes_{R^t} 1 \otimes_{R^s} 1) + (1 \otimes_{R^s} \partial_t(f)\alpha_t^2 \otimes_{R^t} 1 \otimes_{R^s} 1)\\
&= (1 \otimes_{R^s} \alpha_t \otimes_{R^t} P_t(f) \otimes_{R^s} 1) + (1 \otimes_{R^s} 1 \otimes_{R^t} \partial_t(f)\alpha_t^2 \otimes_{R^s} 1),\\
\end{split}
\end{align*}
\noindent and similarly
\begin{align*}
\begin{split}
f \otimes_{R^s} 1 \otimes_{R^t} \alpha_t \otimes_{R^s} 1 = (1 \otimes_{R^s} 1 \otimes_{R^t} P_t(f)\alpha_t \otimes_{R^s} 1) + (1 \otimes_{R^s} \alpha_t \otimes_{R^t} \partial_t(f)\alpha_t \otimes_{R^s} 1).
\end{split}
\end{align*}
\noindent Summing these and combining the ``even'' terms $P_t(f)$ with the ``odd'' terms $\partial_t(f)\alpha_t$, it follows that
\begin{align*}
\begin{split}
f \otimes_{R^s} \alpha_t \otimes_{R^t} 1 \otimes_{R^s} 1 + f \otimes_{R^s} 1 \otimes_{R^t} \alpha_t \otimes_{R^s} 1 &= 1 \otimes_{R^s} \alpha_t \otimes_{R^t} f \otimes_{R^s} 1 + 1 \otimes_{R^s} 1 \otimes_{R^t} f\alpha_t \otimes_{R^s} 1\\
&= 1 \otimes_{R^s} \alpha_t \otimes_{R^t} 1 \otimes_{R^s} f + 1 \otimes_{R^s} 1 \otimes_{R^t} \alpha_t \otimes_{R^s} f.
\end{split}
\end{align*}
\noindent Thus $\psi$ is well-defined, and like $\phi$ it is a monomorphism. Because $\langle 1 \otimes_{R^t} 1\rangle$ and $\langle \alpha_t \otimes_{R^t} 1 + 1 \otimes_{R^t} \alpha_t\rangle$ clearly generate disjoint $(R^s, R^s)$-bimodules, $\phi$ and $\psi$ are themselves disjoint. If we can show
\begin{align*}
\begin{split}
R \otimes_{R^t} R \cong \langle 1 \otimes_{R^t} 1\rangle \oplus \langle \alpha_t \otimes_{R^t} 1 + 1 \otimes_{R^t} \alpha_t\rangle,
\end{split}
\end{align*} %Looking to the images of $\phi$ and $\psi$, we see that
%\begin{align*}
%\begin{split}
%\im(\phi) &= R \otimes_{R^s} R^s \otimes_{R^t} R^s \otimes_{R^s} R,\\
%\im(\psi) &= R \otimes_{R^s} (R^s\alpha_t \otimes_{R^t} R^s \oplus R^s \otimes_{R^t} R^s\alpha_t) \otimes_{R^s} R.
%\end{split}
%\end{align*}
%\noindent Thus these maps are disjoint, as $R^s \otimes_{R^t} R^s$, $R^s\alpha_t \otimes_{R^t} R^s$ and $R^s \otimes_{R^t} R^s\alpha_t$ are pairwise disjoint. In order to show that $B_s \otimes_R B_t \otimes_R B_s \cong B_{sts} \oplus B_s$, we just need to find an $(R^s, R^s)$-bimodule isomorphism
%\begin{align*}
%\begin{split}
%R \otimes_{R^t} R \cong R^s \otimes_{R^t} R^s \oplus R^s\alpha_t \otimes_{R^t} R^s \oplus R^s \otimes_{R^t} R^s\alpha_t.
%\end{split}
%\end{align*}
\noindent then $B_s \otimes_R B_t \otimes_R B_s \cong B_{sts} \oplus B_s$, with a similar computation yielding $B_t \otimes_R B_s \otimes_R B_t \cong B_{sts} \oplus B_t$.\newpage

\noindent To this end, observe that $\alpha_t - (\alpha_s, \alpha_t)\alpha_s = \alpha_t - \frac{1}{2}\alpha_s \in R^s$ and $\alpha_s - \frac{1}{2}\alpha_t \in R^t$, whence
\begin{align*}
\begin{split}
&-\frac{2}{3}\left(\alpha_t - \frac{1}{2}\alpha_s\right) \otimes_{R^t} 1 + \bigg(\alpha_t \otimes_{R^t} 1 + 1 \otimes_{R^t} \alpha_t\bigg) + 1 \otimes_{R^t} -\frac{4}{3}\left(\alpha_t - \frac{1}{2}\alpha_s\right)\\
=\ &\left(\left(\frac{1}{3}\alpha_s - \frac{2}{3}\alpha_t\right) \otimes_{R^t} 1 + \alpha_t \otimes_{R^t} 1\right) + \left(1 \otimes_{R^t} \alpha_t + 1 \otimes_{R^t} \left(\frac{2}{3}\alpha_s - \frac{4}{3}\alpha_t\right)\right)\\
=\ &\left(\frac{1}{3}\alpha_s + \frac{1}{3}\alpha_t\right) \otimes_{R^t} 1 + 1 \otimes_{R^t} \left(\frac{2}{3}\alpha_s - \frac{1}{3}\alpha_t\right)\\
=\ &\alpha_s \otimes_{R^t} 1.
\end{split}
\end{align*}
\noindent From here, it is easy to see that we can obtain all of $R \otimes_{R^t} R$, giving us the desired result.\\[-1.5\baselineskip]
\begin{center}
\rule{0.5\linewidth}{1pt}
\end{center}
\noindent\\[-\baselineskip] All that's left is to clean up a few loose ends. First, we see that $B_{st}$ and $B_{ts}$ are not isomorphic, as
\begin{align*}
\begin{split}
B_s \otimes_R B_t \otimes_R B_s \cong B_{sts} \oplus B_s \not\cong B_{st}(-1) \oplus B_{st}(1) \cong B_s \otimes_R B_s \otimes_R B_t.
\end{split}
\end{align*}
\noindent Note that we have used the fact here that direct sum decompositions are unique. Finally, in order to show that all indecomposables have been exhausted, we claim that
\begin{align*}
\begin{split}
B_{sts} \otimes_R B_s \cong B_s \otimes_R B_{sts} \cong B_{sts}(1) \oplus B_{sts}(-1) \cong B_t \otimes_R B_{sts} \cong B_{sts} \otimes_R B_t.
\end{split}
\end{align*}
\noindent Recall that in \hyperref[RSplitting]{Lemma \ref*{RSplitting}}, we found an isomorphism $R \cong R^s \oplus R^s(-2)$ of $(R^s, R^s)$-bimodules. It is easy to see that this restricts to an isomorphism of $(R^{s,t}, R^s)$-bimodules, whence
\begin{align*}
B_{sts} \otimes_R B_s &\cong (R \otimes_{R^{s,t}} R(3)) \otimes_R (R \otimes_{R^s} R(1))\\
&\cong R \otimes_{R^{s,t}} R \otimes_{R^s} R(4)\\
&\cong R \otimes_{R^{s,t}} (R^s \oplus R^s(-2)) \otimes_{R^s} R(4) \tag*{(\hyperref[RSplitting]{Lemma \ref*{RSplitting}})}\\
&\cong (R \otimes_{R^{s,t}} R(4)) + (R \otimes_{R^W} R(2))\\
&\cong B_{sts}(1) \oplus B_{sts}(-1).
\end{align*}
\noindent The other isomorphisms follow similarly. We have thus shown that we obtain no new indecomposables by tensoring, meaning we are done with our classification. This completes the proof.
\end{proof}\\

\end{document}